\section{Introduction}

インターネット上で経済的価値の移転を安全に管理するという課題は、Bitcoinの登場によって現実味を帯びた。
Bitcoinはデジタル資産の所有権とその移転を分散的に記録する手段を提供し、取引に条件を付与して自動的に実行するという発想──Nick Szaboが提唱した「スマートコントラクト」──を再び注目させた。
Bitcoinは信頼できる第三者への依存を最小化する設計を実現したが、Ethereum等の汎用VMベース環境ではその問題が再燃する可能性が示唆される。
以降、Bitcoin Scriptに代表される単純な条件記述から、Ethereumのようなチューリング完全なステートフル実行環境へと進化が進んだが、その過程で新たな設計上のトレードオフと課題が明らかになった。

本稿は、「トラストレスで分散が必要な特定用途(用途特化型)」にフォーカスしたレイヤー1(L1)チェーンと、そこに適合するスマートコントラクト設計を提案することを目的とする。
特に匿名電子商取引(例:Silk Roadに類する市場)を典型的ユースケースとして取り上げ、匿名市場において顕在化するExit ScamやEscrow(エスクロー)運用といった制度的課題を、暗号技術とComputational Law(計算機化された法手続き)を組み合わせたスマートコントラクトでどのように解決できるかを問い直す。

従来プラットフォームの限界は概ね二つに整理できる。
第一に、Bitcoin Scriptのような表現力の制約は複雑な機能や自律的な金融プロトコルの実装を著しく困難にする点。
第二に、Ethereumに代表される汎用VMベースのアプローチは高い記述力を実現した一方で、監査の必然性、利用者が署名時に内容を読み解きにくいことによる第三者依存、資産の所有権がスマートコントラクト側へ移る運用、さらにアップグレード可能性に伴うトラスト要請といった問題を生んだ。
これらは第三者依存を警戒する観点を再燃させ、実運用では盲信的な署名や、DAO事件に見られるような重大な安全インシデントを招く温床となっている。

以上の背景を踏まえ、本稿では以下の見取り図を提示する。
まず、スマートコントラクトを「汎用的なスマートプログラム」ではなく、用途ごとに厳選された雛形(template)と強力なロック機構を組み合わせたシンプルかつ説明可能な契約形式として再定義する。
そのうえで、匿名電子商取引という実用的ユースケースにおいて、Exit Scamやエスクローの失敗を防ぐためのトランザクションモデルを設計する。
設計方針としては、(1)ユーザーがトランザクション内容を解読可能であること、(2)グローバルステートへの過度な依存を避けることを要求要件とし、その手段としてUTXOベースのロック指向コントラクトで資産移動を扱う。

本稿の主な貢献は次の三点である。
第一に、既存のBitcoin ScriptとEthereum的なVM実装の中間を志向する、可読性と表現力のバランスを取ったスマートコントラクトデータ構造およびトランザクションモデルを提示する。
第二に、匿名市場におけるExit ScamとEscrow問題に対して、UTXOベースのスマートコントラクトを設計する。
第三に、提案モデルが既存プラットフォーム(Bitcoin, Ethereum)および関連研究(CardanoやAztecなどのUTXO派・プライバシー技術を含む)のどの点を補完・改善するかを示すことで、用途特化型L1の有効性を議論する。

方法論としては、提案するテンプレート型スマートコントラクトと(e)UTXO実行モデルを形式的に定義し、資金保全・権限安全・活性(デッドロックなし)といった不変条件を仕様レベルで与える。
加えて、監査可能性(複雑性の上限)、検証量、状態増大の見積を通じて、汎用VM方式とのトレードオフを整理する。
必要に応じて、最小の参照実装(トランザクション検証器レベル)またはシミュレーションにより実現可能性を補助的に検討する。

本稿の貢献は次の通りである。
第一に、スマートコントラクトをテンプレート型の条件付き資産移転として再定義し、許容条件(admissibility)と不変条件(資金保全・権限安全・活性)に基づく形式的枠組みを提示する。
第二に、汎用VMに依存せず、(e)UTXO上でテンプレートを実行するトランザクションモデル(ロック指向の局所状態・テンプレート適用・検証規則)を提案する。
第三に、焦点ケースとしてのカストディ型エスクローの失敗モードに対し、運営の単独カストディを避ける設計をテンプレートとして与え、既存の緩和策との差分(何が運用裁量として残り、何を検証可能な拘束へ移送するか)を明示する。

以降の構成は以下の通りである。
第2章ではスマートコントラクト概念と既存実装様式(制約スクリプトと汎用VM)を整理し、第3章では焦点ケースのモデル化(アクター・ワークフロー・失敗モード)と要件定義を行う。
第4章でテンプレート型スマートコントラクトの形式化を与え、第5章で(e)UTXO上の実行モデルとテンプレート群を提示する。
第6章では安全性・監査可能性・コスト見積に基づく比較と議論を行い、第7章で結論と今後の課題を述べる。
