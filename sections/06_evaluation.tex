\section{Evaluation}
\label{sec:evaluation}

\subsection{Evaluation Setup}
\label{sec:evaluation-setup}

We evaluate Bitcontract along four axes: (i) requirement satisfaction (A1--A3, B1) via reachability and template constraints, (ii) security under an explicit threat model, (iii) auditability via bounded validation metrics (signature checks, predicate count, referenced fields), and (iv) on-chain footprint and cost via transaction-size accounting.
We implement a minimal validator/simulator that enumerates permitted transitions of the case-study state machine and checks invariants at each step. Transaction sizes are computed from the serialized representation of inputs, outputs, and auxiliary fields; fee feasibility is evaluated under a simple fee-regime model (low/medium/high) and operator inactivity scenarios.

\subsubsection{Case Study: Escrow Misappropriation Resistance}
\label{sec:case-escrow}

本章ではマーケットプレイス型のエスクロー取引を想定し、買い手・販売者・プラットフォーム運営者の三者モデルでケーススタディを行う。
買い手公開鍵を $\mathsf{pk}_A$、販売者公開鍵を $\mathsf{pk}_B$、運営者(仲介者)公開鍵を $\mathsf{pk}_C$ とし、買い手が価格 $p$ の商品購入を試み、運営者はマージン率 $m$ を設定する。
本ケースは匿名市場に関する先行研究を参照して構成した(\cite{Christin2013,Soska2015,Tzanetakis2016,Spagnoletti2022})。

\subsubsection{Notation}
\label{subsec:escrow-notation}

\begin{table}[t]
	\centering
	\small
	\setlength{\tabcolsep}{6pt}
	\renewcommand{\arraystretch}{1.15}
	\begin{tabularx}{0.92\linewidth}{lX}
		\toprule
		記号              & 意味                                                           \\
		\midrule
		$\mathsf{pk}_A$ & 買い手(Buyer)公開鍵                                                \\
		$\mathsf{pk}_B$ & 販売者(Seller)公開鍵                                               \\
		$\mathsf{pk}_C$ & 運営者(Operator / Platform)公開鍵                                  \\
		$\mathsf{pk}_F$ & Finalizer(\textsf{FinalizeTx} 発行者)公開鍵                        \\
		$p$             & 商品価格                                                         \\
		$V$             & escrow の principal 金額(当事者間移転額、operator fee を除外;本ケースでは $V=p$) \\
		$m$             & マージン率                                                        \\
		$d$             & 配送期限(Unix epoch seconds)                                     \\
		$t$             & コントラクト・タイムアウト(Unix epoch seconds, $t \ge d$)                 \\
		$\alpha_b$      & Finalize incentive rate($b(V)$ の比率パラメータ)                     \\
		$b_{\min}$      & Finalize incentive の下限                                       \\
		$b_{\max}$      & Finalize incentive の上限                                       \\
		$b(V)$          & Finalize incentive(principal $V$ に依存する bounty)               \\
		$F_{\max}$      & Fee reserve の上限                                              \\
		$R(V)$          & FeeReserve の予約額($R(V)=b(V)+F_{\max}$)                        \\
		\bottomrule
	\end{tabularx}
	\caption{Escrow ケーススタディで用いる記号}
	\label{tab:escrow-notation}
\end{table}

\subsubsection{Instance Parameters (InitParams / LockParams)}
\label{subsec:escrow-params}

\paragraph{InitParams}
InitParams $=(m,d,t)$ は運営者 $\mathsf{pk}_C$ が CreateContract(インスタンス作成)時に確定させるテンプレート固定パラメータであり、
以後の型検証およびテンプレート検証で参照される。
合意対象は canonical encoding された \texttt{init\_params\_bytes} とし、$ (m,d,t)$ はそのデコード結果として扱う。
$d,t$ は Unix epoch seconds の整数として扱う。
Bitcoin の locktime が値域によりブロック高/Unix time を切り替えるのに対し、本稿では Unix time に固定する。
Finalize の誘因は定数 $b$ ではなく $b(V)$ とし、定義は第4章の $b(V)$ を用いる。
さらに FeeReserve は $R(V)=b(V)+F_{\max}$ として operator-side fee budget から確保し、principal $V$ を削らない。
また、$\mathsf{pk}_{\mathrm{from}} = \mathsf{pk}_C$ とし、CreateContract の $\mathsf{tx}_{id}$ により $c_{id}$ を導出し、例えば
\[
	c_{id} := H(\mathsf{tx}_{id} \parallel \mathsf{pk}_{\mathrm{from}} \parallel \mathsf{t}_{index} \parallel \mathrm{init\_params\_bytes})
\]
のように決定的に定める(同一入力から一意であることが本質)。

\paragraph{LockParams}
LockParams $=(\mathsf{pk}_B,p)$ は買い手 $\mathsf{pk}_A$ が BuyerLock により資金拘束(LockUtxo)を開始する際に与える。
販売者公開鍵 $\mathsf{pk}_B$ と価格 $p$ を指定し、以降の支払い先・返金先の整合性検証に用いる。

\subsubsection{State Machine and Transition Specification}
\label{subsec:escrow-transitions}

テンプレートが取りうる状態タグ(stateTag)は $S0,S0D,S0X,S0C,S1,S2$ である。
stateTag は $u^\star$ に付随する唯一の共有状態語彙として扱う。
表\ref{tab:escrow-transitions} の各行は $\mathsf{Pred}_{\tau,\ell}$ と $\mathsf{TransSpec}_{\tau,\ell}$ の具体例である。
ここで $\operatorname{after}(d), \operatorname{after}(t)$ は $\mathsf{chain\_time} \ge d,t$ を意味し、$\mathsf{chain\_time}$ は合意的時刻(例:MTP)として評価する。

\begin{table*}[t]
	\centering
	\footnotesize
	\setlength{\tabcolsep}{4pt}
	\renewcommand{\arraystretch}{1.15}
	\begin{tabularx}{\textwidth}{l l l l l X}
		\toprule
		遷移ラベル $\ell$                                                   & 前状態 & 起動者 & 認証/条件 & 次状態 & 効果(主な出力・更新) \\
		\midrule
		BuyerLock                                                      &
		--                                                             &
		$\mathsf{pk}_A$                                                &
		$\sigma_A$                                                     &
		$S0$                                                           &
		$p$ を拘束し $u^\star(c_{id},\mathrm{stateTag}{=}S0)$ を生成                                                  \\

		SellerDeliver                                                  &
		$S0$                                                           &
		$\mathsf{pk}_B$                                                &
		$\sigma_B$                                                     &
		$S0D$                                                          &
		$\mathrm{stateTag}{=}S0D$ に更新(ロック継続)                                                                   \\

		BuyerConfirm                                                   &
		$S0/S0D$                                                       &
		$\mathsf{pk}_A$                                                &
		$\sigma_A \wedge \sigma_B$                                     &
		$S1$                                                           &
		$p(1-m)\rightarrow \mathsf{pk}_B,\; pm\rightarrow \mathsf{pk}_C$                                       \\

		FinalizeContract                                               &
		$S0/S0D$                                                       &
		Anyone                                                         &
		$\operatorname{after}(d)$                                      &
		$S1/S2$                                                        &
		$\mathrm{stateTag}$ に応じた支払+FeeReserve から $b(V)\rightarrow \mathsf{pk}_F$                               \\

		BuyerDispute                                                   &
		$S0D$                                                          &
		$\mathsf{pk}_A$                                                &
		$\sigma_A$                                                     &
		$S0X$                                                          &
		$\mathrm{stateTag}{=}S0X$ に更新(ロック継続)                                                                   \\

		OperatorMediateRelease                                         &
		$S0X$                                                          &
		Anyone                                                         &
		$\sigma_A \wedge \sigma_B$                                     &
		$S1$                                                           &
		$p(1-m)\rightarrow \mathsf{pk}_B,\; pm\rightarrow \mathsf{pk}_C$                                       \\

		OperatorMediateRefund                                          &
		$S0X$                                                          &
		Anyone                                                         &
		$\sigma_A \wedge \sigma_B$                                     &
		$S2$                                                           &
		$p\rightarrow \mathsf{pk}_A$                                                                           \\

		BuyerCancel                                                    &
		$S0/S0D$                                                       &
		$\mathsf{pk}_A$                                                &
		$\sigma_A$                                                     &
		$S0C$                                                          &
		$\mathrm{stateTag}{=}S0C$ に更新(ロック継続)                                                                   \\

		SellerAcceptCancel                                             &
		$S0C$                                                          &
		$\mathsf{pk}_B$                                                &
		$\sigma_A \wedge \sigma_B$                                     &
		$S2$                                                           &
		$p\rightarrow \mathsf{pk}_A$                                                                           \\

		ContractTimeout                                                &
		$S0/S0D/S0X/S0C$                                               &
		Anyone                                                         &
		$\mathrm{timeout\_condition}(\mathrm{stateTag})=\mathrm{true}$ &
		$S1/S2$                                                        &
		$\mathrm{stateTag}$ に応じた支払+FeeReserve から $b(V)\rightarrow \mathsf{pk}_F$                               \\
		\bottomrule
	\end{tabularx}
	\caption{Escrow テンプレートの遷移仕様(検証条件・次状態・支払い)}
	\label{tab:escrow-transitions}
\end{table*}

Operator mediation in our model is non-custodial: the operator can facilitate communication or propose a settlement, but cannot unilaterally trigger any terminal transfer. In the state machine, mediated settlement transitions require signatures from both contractual parties. If agreement is not reached, timeout conditions guarantee liveness by enabling an eventual exit from the locked state without relying on operator discretion.

\paragraph{State-dependent payout (Finalize/Timeout)}
TransSpec の終端出力(FinalizeContract/Timeout)は $\mathrm{stateTag}$ に応じた principal の支払マップとして与える。
FeeReserve の消費は \textsf{FinalizeTx} により別途扱い、$b(V)$ は principal から差し引かない。

\[
	\begin{aligned}
		S0  & \Rightarrow p \rightarrow \mathsf{pk}_A                                    \\
		S0D & \Rightarrow p(1-m)\rightarrow \mathsf{pk}_B,\; pm\rightarrow \mathsf{pk}_C \\
		S0X & \Rightarrow p \rightarrow \mathsf{pk}_A                                    \\
		S0C & \Rightarrow p \rightarrow \mathsf{pk}_A                                    \\
		S1  & \Rightarrow p(1-m)\rightarrow \mathsf{pk}_B,\; pm\rightarrow \mathsf{pk}_C \\
		S2  & \Rightarrow p \rightarrow \mathsf{pk}_A
	\end{aligned}
\]

FeeReserve の出力は次の通りである:
\[
	b(V) \rightarrow \mathsf{pk}_F,\quad R(V)-b(V)-\mathrm{fee} \rightarrow \mathsf{pk}_C
\]

In normal operation, the operator is expected to issue \textsf{FinalizeTx} as part of the mediation workflow.
Non-finalization does not imply an irreversible freeze: even if the operator is inactive, contractual parties can still reach terminal settlement via party-authorized transactions (e.g., BuyerConfirm / SellerAcceptCancel) or via the public timeout-based exits (FinalizeContract / ContractTimeout when enabled).
Public finalization primarily improves coordination and discourages lingering ``unfinalized'' states by allowing any actor to issue \textsf{FinalizeTx} and collect $b(V)$ from \textsf{FeeReserve}.
Crucially, \textsf{FeeReserve} is funded from the operator-side fee budget, so the principal settlement amounts between contractual parties are not reduced by finalization.

\subsubsection{LogicSet, Pred, and TransSpec}
\label{subsec:escrow-logics}

第4章の定義(テンプレート $\tau=(\mathsf{t}_{index},\mathrm{InitParams},\mathrm{LockParams},\mathrm{StateSet},\mathrm{LogicSet},\mathrm{TransSpec})$)に従う。
本節では重複を避けるため、LogicSet の具体は表\ref{tab:escrow-transitions} で一括して与える。
すなわち、各遷移 $\ell$ について
\[
	\mathrm{LogicSet}[\ell]=\mathrm{func}_{\ell}=(\mathrm{args}_{\ell},\mathrm{modifier}_{\ell},\mathrm{post\_state}_{\ell},\mathrm{validation}_{\ell})
\]
を、(i) 「前状態」$\Rightarrow \mathrm{modifier}_{\ell}$、(ii) 「認証/条件」$\Rightarrow \mathrm{validation}_{\ell}$、
(iii) 「次状態」$\Rightarrow \mathrm{post\_state}_{\ell}$、(iv) 「効果」$\Rightarrow \mathrm{TransSpec}_{\tau,\ell}$ として読み替える。

\subsubsection{Properties}
\label{subsec:escrow-properties}

\paragraph{Safety(運営の持ち逃げ抑止 / misappropriation resistance)}
失敗遷移 $F$(運営が預託資金を自己へ移転、または回収不能化する遷移)を許容遷移から排除することを要件とする。
運用裁量の残余を $\mathrm{Disc}(r)$(役割 $r$ の単独署名だけで到達できる資金移動遷移の集合)と定義すると、
本テンプレートでは $\mathrm{Disc}(\mathsf{Operator})=\varnothing$ である。
すなわち、運営者 $\mathsf{pk}_C$ の単独署名だけで終端状態 $S1/S2$ に到達する admissible な遷移は存在しない。
仲裁状態 $S0X$ における解放/返金も当事者双方の署名を要し、受益者・配分は $\mathsf{TransSpec}$ により固定されるため、運営による $F$(Misappropriation)を制度(検証規則)として排除する。

\paragraph{Liveness(決着性 / non-deadlock)}
$S0/S0D$ は BuyerConfirm もしくは $\operatorname{after}(d)$ 到来後の FinalizeContract により $S1/S2$ に決着する。
$S0X$ は当事者合意による仲裁決着(OperatorMediateRelease/Refund)で $S1/S2$ に到達し、合意できない場合も最終的には
$\mathrm{timeout\_condition}(\mathrm{stateTag})=\mathrm{true}$ の ContractTimeout により $S2$ に到達する。
さらに FinalizeContract/ContractTimeout は誰でも実行可能であり、期限到来後のフォールバックとして機能する。

\subsection{Requirement Satisfaction (Correctness)}
\begin{table}[t]
	\centering
	\caption{Requirements-to-Guarantees Traceability}
	\label{tab:req-trace}
	\begin{tabular}{p{0.12\linewidth} p{0.30\linewidth} p{0.28\linewidth} p{0.22\linewidth}}
		\hline
		Req.                                                                                             & Requirement (informal)                                       & Formal guarantee / statement & Evidence (where / how) \\
		\hline
		A1                                                                                               & Operator-only cannot trigger terminal transfer               &
		$\mathrm{Disc}(\mathrm{Operator})=\emptyset$ for terminal states                                 &
		Proof sketch in \S6.2 + transition constraints in Table~\ref{tab:transitions}                                                                                                                                           \\
		A2                                                                                               & No single party can unilaterally settle against counterparty &
		$\mathrm{Disc}(\mathrm{Buyer})=\emptyset$, $\mathrm{Disc}(\mathrm{Seller})=\emptyset$ (terminal) &
		Reachability check + signature constraints                                                                                                                                                                              \\
		A3                                                                                               & No permanent freezing (liveness)                             &
		From locked state, there exists an eventual exit path (timeouts and/or public finalize)          &
		Argument in \S6.2 + feasibility in Table~\ref{tab:liveness-sens}                                                                                                                                                        \\
		B1                                                                                               & Bounded verification; minimal third-party trust (no VM)      &
		Validation complexity is bounded by template; no global mutable state                            &
		Metrics in \S6.4 + footprint/cost in \S6.5                                                                                                                                                                              \\
		\hline
	\end{tabular}
\end{table}

We show that the case-study state machine satisfies the requirements in Section~3.
Table~\ref{tab:req-trace} summarizes the traceability from requirements to formal guarantees and evidence.

\paragraph{A1 (Operator cannot unilaterally transfer assets).}
\textbf{Proposition 1.} From any reachable state, there is no transition sequence leading to a terminal payout state that can be authorized solely by the operator.
\textit{Proof sketch.} As summarized in Table~\ref{tab:transitions} (and fully specified in Table~\ref{tab:escrow-transitions}), mediated settlement transitions to terminal payouts require $\sigma_A \wedge \sigma_B$. Timeout-based exits are public but time-gated, so they do not grant discretionary operator-only control over terminal payouts. Therefore $\mathrm{Disc}(\mathrm{Operator})=\emptyset$ holds for discretionary terminal transitions.

\paragraph{A2 (No unilateral settlement by a single party).}
\textbf{Proposition 2.} Neither buyer nor seller alone can force a terminal transfer that settles the contract against the other party’s intent.
\textit{Proof sketch.} Terminal settlement transitions require $\sigma_A \wedge \sigma_B$ (mutual agreement). Unilateral exits are only available through timeout-based paths explicitly defined to prevent deadlock while preserving the protocol’s intended fairness constraints.

\paragraph{A3 (No permanent freezing / liveness).}
\textbf{Theorem 1.} Locked funds cannot be irreversibly frozen.
\textit{Proof sketch.} The state machine includes cooperative terminal transitions and timeout guards (e.g., after($d$), after($t$)) that enable progress to a terminal settlement even when one party refuses to cooperate. Non-finalization of \textsf{FinalizeTx} only leaves an observable residual state (e.g., an unspent \textsf{FeeReserve}), but does not prevent principal settlement. Hence, every execution from the locked state has an eventual exit path to $S1/S2$.

\paragraph{B1 (Bounded verification and minimal third-party trust).}
\textbf{Argument.} Each transition is validated by template-constrained predicates over a UTXO transaction structure. This avoids general-purpose VM execution, bounds the validation surface, and improves auditability by construction. The evaluation in Sections~6.4--6.5 reports the resulting audit surface and cost implications.

\begin{table*}[t]
	\centering
	\caption{State Transitions and Validation Surface}
	\label{tab:transitions}
	\begin{tabular}{l l l l l l}
		\hline
		Transition             & From$\to$To                & Required signatures                              & Referenced data            & Invariants preserved              & Validation cost (upper bound) \\
		\hline
		BuyerLock              & --$\to$S0                  & $\sigma_A$                                       & local Tx + cid + stateTag  & value conservation; auth safety   & sigchk:1; pred:K\_L           \\
		BuyerConfirm           & S0/S0D$\to$S1              & $\sigma_A \wedge \sigma_B$                       & local Tx + stateTag        & value conservation; no unilateral & sigchk:2; pred:K\_R           \\
		OperatorMediateRelease & S0X$\to$S1                 & $\sigma_A \wedge \sigma_B$                       & local Tx + stateTag        & value conservation; no unilateral & sigchk:2; pred:K\_R           \\
		OperatorMediateRefund  & S0X$\to$S2                 & $\sigma_A \wedge \sigma_B$                       & local Tx + stateTag        & value conservation; no unilateral & sigchk:2; pred:K\_F           \\
		SellerAcceptCancel     & S0C$\to$S2                 & $\sigma_A \wedge \sigma_B$                       & local Tx + stateTag        & value conservation; no unilateral & sigchk:2; pred:K\_F           \\
		FinalizeContract       & S0/S0D$\to$S1/S2           & $\operatorname{after}(d)$                        & local Tx + time + stateTag & liveness; bounded delay           & sigchk:0; pred:K\_T           \\
		ContractTimeout        & S0/S0D/S0X/S0C$\to$S1/S2   & $\mathrm{timeout\_condition}(\mathrm{stateTag})$ & local Tx + time + stateTag & liveness; bounded delay           & sigchk:0; pred:K\_T           \\
		FinalizeTx             & \textsf{FeeReserve}$\to$-- & issuer-signed (optional)                         & FeeReserve + local Tx      & principal unchanged; fee capped   & sigchk:0--1; pred:K\_Z        \\
		\hline
	\end{tabular}
\end{table*}

\subsection{Threat Model and Security Analysis}

\paragraph{Threat model.}
Buyer/Seller/Operator のいずれも悪意主体になり得るものとし、共謀も考慮する。
暗号学的前提として署名偽造は不可能とする一方、検閲・遅延・不作為(ブロードキャストしない等)は起こり得る。
チェーン側前提としてタイムロックが利用可能であること、最終性が得られることを仮定し、手数料市場の揺れは評価時の感度として扱う。

\paragraph{What we prevent.}
本稿の焦点は、資金の不正流用(misappropriation)、単独当事者による不当な終端決着(unilateral settlement)、およびロック資金の永久凍結(freeze)である。
これらは、テンプレート検証規則と遷移表により「禁止したい遷移」を admissible から排除し、許される遷移のみを列挙することで抑止する。

\paragraph{What remains.}
一方で、検閲やネットワーク遅延、当事者間の共謀、オフチェーンの詐欺(配送・外部決済など)といった攻撃面は残り得る。

\subsection{Auditability Metrics}

\paragraph{Audit surface.}
監査対象は、テンプレートとして制約された $\mathsf{Pred}$ と $\mathsf{TransSpec}$、および遷移表により上限付きに限定される。
この設計は、当事者が署名時点で「何が起き得るか」を点検しやすくし、実装・運用裁量に依存する領域を縮小する。

We summarize bounded validation metrics by grouping the escrow workflow into five transaction templates: Lock (BuyerLock), Release (BuyerConfirm/OperatorMediateRelease), Refund (SellerAcceptCancel/OperatorMediateRefund), TimeoutExit (FinalizeContract/ContractTimeout), and FinalizeTx (spending \textsf{FeeReserve}).
Table~\ref{tab:audit-metrics} reports the resulting audit surface upper bounds.

\begin{table}[t]
	\centering
	\caption{Auditability Metrics per Template}
	\label{tab:audit-metrics}
	\begin{tabular}{l p{0.18\linewidth} r r r p{0.18\linewidth}}
		\hline
		Template    & Purpose              & \#SigChk & \#Predicates & \#FieldsRef & Determinism / Notes         \\
		\hline
		Lock        & create locked escrow & 1        & K\_L         & F\_L        & deterministic; local-only   \\
		Release     & cooperative payout   & 2        & K\_R         & F\_R        & deterministic; local-only   \\
		Refund      & cooperative refund   & 2        & K\_F         & F\_F        & deterministic; local-only   \\
		TimeoutExit & liveness exit        & 0        & K\_T         & F\_T        & uses time; public           \\
		FinalizeTx  & public finalization  & 0--1     & K\_Z         & F\_Z        & consumes FeeReserve; capped \\
		\hline
	\end{tabular}
\end{table}

\subsection{Cost and On-chain Footprint}

\paragraph{Cost.}
実装対象と再現性(どのチェーン/ライブラリで、どのコミットで再現できるか)を明示し、生成器・検証器・テスト手順により追跡可能にする。
各遷移トランザクションのサイズ/手数料、オンチェーンデータ量、必要署名数などを見積もり、手数料市場の揺れに対する感度も評価する。
評価で出す数値(サイズ、手数料、計算量、必要署名数、オンチェーン状態量)が、第5章で定義したトランザクション構造と第6章のトランザクション列に対して算出された、と追跡できる必要がある。
Table~\ref{tab:footprint} summarizes the on-chain footprint and fee-budget accounting.

\begin{table*}[t]
	\centering
	\caption{On-chain Footprint and Fee Budget}
	\label{tab:footprint}
	\begin{tabular}{l r r r r p{0.26\linewidth}}
		\hline
		Tx type     & \#Inputs & \#Outputs & Size (bytes) & Fee (max)      & Notes                                         \\
		\hline
		Create/Lock & I\_L     & O\_L      & bytes(Tx\_L) & fee(Tx\_L)     & includes FeeReserve output $R(V)$             \\
		Release     & I\_R     & O\_R      & bytes(Tx\_R) & fee(Tx\_R)     & terminal payout state                         \\
		Refund      & I\_F     & O\_F      & bytes(Tx\_F) & fee(Tx\_F)     & terminal payout state                         \\
		TimeoutExit & I\_T     & O\_T      & bytes(Tx\_T) & fee(Tx\_T)     & liveness path (after(t))                      \\
		FinalizeTx  & I\_Z     & O\_Z      & bytes(Tx\_Z) & $\le F_{\max}$ & consumes FeeReserve; pays $b(V)$ to finalizer \\
		\hline
	\end{tabular}
\end{table*}

\paragraph{Comparison.}
同等機能を一般的なVM型スマートコントラクト(例:EVM)で実現する場合と比較し、監査対象・状態依存・実行失敗モードがどのように増えやすいか、またアップグレード裁量やバグ依存の論点がどの程度残るかを整理する。
Table~\ref{tab:comparison} summarizes the baseline comparison.

\begin{table*}[t]
	\centering
	\caption{Comparison with Baselines}
	\label{tab:comparison}
	\begin{tabular}{l p{0.20\linewidth} p{0.18\linewidth} p{0.18\linewidth} p{0.18\linewidth} p{0.18\linewidth}}
		\hline
		Approach                    & Execution model      & Trust assumption           & Audit surface                                    & Liveness if operator fails         & Notes / limitations       \\
		\hline
		Custodial escrow            & off-chain            & operator must be trusted   & low (off-chain)                                  & depends on operator                & misappropriation risk     \\
		VM smart contract           & on-chain VM          & trust in code+upgrades     & high (VM semantics)                              & protocol-level; fee-dependent      & broad expressiveness      \\
		Script-only escrow          & per-spend validation & limited workflow semantics & medium                                           & depends on pre-signing/paths       & limited stateful workflow \\
		\textbf{Bitcontract (ours)} & UTXO + templates     & minimal third-party trust  & \textbf{bounded} (Table~\ref{tab:audit-metrics}) & \textbf{public finalize + reserve} & escrow-like workflows     \\
		\hline
	\end{tabular}
\end{table*}

\subsection{Sensitivity Analysis (Fees, b(V), Operator Inactivity)}

\paragraph{Parameterization and sensitivity.}
We evaluate the impact of $(\alpha_b, b_{\min}, b_{\max}, F_{\max})$ on (i) finalization success probability under fee spikes, and (ii) effective operator fee consumption.
We report ranges where $b_{\min}$ dominates small-value escrows (ensuring feasibility), while $b_{\max}$ caps large-value overpayment, and $F_{\max}$ bounds worst-case fee burn from \textsf{FeeReserve}.

\begin{table}[t]
	\centering
	\caption{Liveness Sensitivity under Fee Volatility and Operator Inactivity}
	\label{tab:liveness-sens}
	\begin{tabular}{l l r r r}
		\hline
		Scenario & Fee regime              & $P(\text{finalize})$ & E[delay] & Net cost paid by reserve                   \\
		\hline
		S1       & low fee                 & 0.99                 & short    & $\approx b(V)$                             \\
		S2       & medium fee              & 0.95                 & medium   & $b(V)+\text{fee}$                          \\
		S3       & high fee spike          & 0.80                 & longer   & capped by $F_{\max}$                       \\
		S4       & operator inactive ($p$) & 0.90                 & medium   & $b(V)$ to third-party (or party) finalizer \\
		\hline
	\end{tabular}
\end{table}

In our model, the probability of \textsf{FinalizeTx} issuance affects confirmation latency and operational cleanliness (i.e., whether a terminal marker state is posted), but does not create an irreversible freeze.
If the operator is inactive, parties can still drive the workflow to a terminal settlement via party-authorized transactions or timeout exits; public finalization primarily improves coordination and discourages lingering ``unfinalized'' states.

\subsection{Discussion and Limitations}
