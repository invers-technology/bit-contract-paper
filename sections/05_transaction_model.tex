o
\section{Transaction Model}
本章では、UTXOを用いたスマートコントラクトの実現に関するトランザクションを論じる。
4章で与えたテンプレート/インスタンス/Locked UTXO および admissibility の定義を前提に、トランザクションの型(classes)として作成・実行・確定の流れを整理する。
本章で扱うトランザクションは、CreateContract、LockUtxo、ExecuteContract($tx_\ell$)、FinalizeContract($tx_{\ell_{\mathrm{final}}}$)の4種類である。
トランザクションの一般形は第4章(Formalization)の Definition 3 に従う。以降は各“型”ごとに、Definition 3 の $\mathsf{out}/\mathsf{aux}$ に課す追加制約(marker、含めるパラメータ、$\mathsf{c}_{id}$ 導出など)のみを列挙する。
本章の各遷移 $\ell$ は、stateTag付きの Locked UTXO を入力に取り、$\mathsf{TransSpec}$ が指定する (i) 次の Locked UTXO への更新 または (ii) 解放(S1/S2)の出力 を生成する“トランザクション型”である。
以降の具体化では、Escrow のテンプレート index を $\mathsf{t}_{index}=1$ と固定する。以下の表記は本章のトランザクションモデルに従う。
\begin{enumerate}
	\item オペレーターが自らのアドレス $\Oper$ とパーセンテージのマージン率 $m$ を含めた CreateContract Transaction を送信する。
	\item バイヤーが購入する商品の価格 $p$ と販売者のアドレス $\Seller$ を含めた LockUtxo Transaction を送信する。
	\item 2 で参照した $\mathsf{c}_{id}$ に対して ExecuteContract Transaction($tx_\ell$)を送信し、処理を行う。
\end{enumerate}

\subsection{Creation}
Definition 9 の CreateContract に対応する。CreateContract は $\mathsf{out}$ に $\mathsf{to}=0x0$ の出力(marker/burn)を含み、$\mathsf{aux}$ に $\mathsf{t}_{index}$ と $\mathsf{InitParams}$ を含む。さらに
\begin{equation}
	\mathsf{TypeCheck}(\mathsf{InitParams}, \mathsf{ArgSchema}[\mathsf{t}_{index}])=\mathsf{true}
\end{equation}
を満たし、$\mathsf{c}_{id}$ は
\begin{equation}
	\mathsf{c}_{id} := H(\mathsf{addr}_{\mathrm{deployer}} \parallel \mathsf{t}_{index} \parallel \mathsf{InitParams})
\end{equation}
により決定的に導出される。ここで $\mathsf{ArgSchema}[\mathsf{t}_{index}]$ はテンプレートの Constructor 引数スキーマである。

\subsection{LockUtxo}
Definition 9 の LockUtxo に対応する。通常 UTXO $u$ を消費し、Locked UTXO $u^\star$ を出力する。
出力には $(\mathsf{c}_{id}, \mathsf{stateTag}=s_{\mathrm{init}})$ が付与される。
CreateContract と LockUtxo は実装上同一Txに畳めるが、概念上は分離して記述する。

\subsection{ExecuteContract}
Locked UTXO $u^\star$ を入力として消費するトランザクション ExecuteContract $tx_\ell$ を指す。ここで $\ell$ は遷移ラベルである。
admissible 判定は第4章の $\mathsf{Pred}_{\tau,\ell}$ と $\mathsf{TransSpec}_{\tau,\ell}$ に従う。

継続遷移であれば $\mathsf{out}(tx)$ は次状態の $u^{\star\prime}(\dots,\mathsf{c}_{id}, \mathsf{stateTag}=s')$ を含み、終端遷移であれば受益者への通常 UTXO とお釣り等のみを含む。

\subsection{FinalizeContract}
Definition 11 で導入した確定遷移 FinalizeContract に対応する。
期限経過 $\operatorname{after}(d)$ や $\operatorname{after}(t)$ に対応する遷移ラベル $\ell_{\mathrm{final}}$ を与え、admissibility と同じ枠組みで検証可能にする。
