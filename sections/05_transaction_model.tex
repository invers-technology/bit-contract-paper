\section{Transaction Model}
本章では、$u$ を用いたスマートコントラクトの実現に関する $tx$ を論じる。
4章で与えたテンプレート/インスタンス/$u^\star$ および admissibility の定義を前提に、$tx$ の型(classes)として作成・実行・確定の流れを整理する。
まず、$tx$ を \textbf{Transfer Transaction} と \textbf{Contract Transaction} に二分する。
Transfer Transaction はベースレイヤの通常価値移転であり、本稿の追加検証の対象外(前提)とする。
Contract Transaction はテンプレート参照・$\mathsf{c}_{id}$・$\mathsf{stateTag}$・$\mathsf{TransSpec}$ により検証される契約系遷移である。
Contract Transaction の型は \textbf{CreateContract}(\texttt{tx\_type=CREATE})、\textbf{LockUtxo}(\texttt{tx\_type=LOCK})、\textbf{ExecuteContract}(\texttt{tx\_type=EXECUTE})、\textbf{FinalizeContract}(\texttt{tx\_type=FINALIZE})の4種類である。
典型的なライフサイクルは
\[
	\texttt{Create} \rightarrow \texttt{Lock} \rightarrow \texttt{Execute}^{\ast} \rightarrow (\texttt{Finalize} / \texttt{Timeout} / \texttt{Release})
\]
であり、$\texttt{Execute}^{\ast}$ は 0 回以上の遷移を表す。
$tx$ の一般形は第4章(Formalization)の Definition 3 に従う。
以降は各“型”ごとに、Contract Transaction 共通ヘッダに基づく $\mathsf{aux}$ と $\mathsf{out}$ の追加制約のみを列挙する。
$tx$ 型の判定は $\mathsf{in}/\mathsf{out}$ の有無などの否定条件に依存せず、\texttt{tx\_type} によって一意に定める。
本章の各遷移 $\ell$ は、$\mathsf{stateTag}$ 付きの $u^\star$ を入力に取り、$\mathsf{TransSpec}$ が指定する (i) 次の $u^{\star\prime}$ への更新 または (ii) 解放($S1/S2$)の出力 を生成する Contract Transaction の遷移である。
以降の具体化では、Escrow のテンプレート $\mathsf{t}_{index}=1$ と固定する。
以下の表記は本章の $tx$ モデルに従う。
\begin{enumerate}
	\item オペレーター公開鍵 $\mathsf{pk}_C$ が、マージン率 $m$ を含めた CreateContract $tx$ を送信する。
	\item 買い手公開鍵 $\mathsf{pk}_A$ が、価格 $p$ と販売者公開鍵 $\mathsf{pk}_B$ を含めた LockUtxo $tx$ を送信する。
	\item 2 で参照した $\mathsf{c}_{id}$ に対して ExecuteContract $tx$($tx_\ell$)を送信し、処理を行う。
\end{enumerate}

\subsection{Contract Transaction 判定(\texttt{tx\_type} による分類)}
Contract Transaction の分類は $\mathsf{aux}$ 内の \texttt{tx\_type} により一意に定まり、$\mathsf{in}/\mathsf{out}$ の形状から推測しない。
特に $u^\star$ を入力に含む $tx$ は、未知 \texttt{tx\_type}/非canonical \texttt{aux}/必須フィールド欠落の場合は invalid とする(詳細は第4章の規範定義に従う)。
Contract Transaction の型判定は $\mathsf{aux}$ に含まれる \texttt{tx\_type} に基づき、$\mathsf{in}/\mathsf{out}$ の欠如などの形推測では分類しない。
判定手順は次である。
\begin{quote}
	\small
	\begin{verbatim}
Parse aux -> fields
if fields.tx_type not in {CREATE, LOCK, EXECUTE, FINALIZE}: return NotContract Transaction
if not HasRequiredFields(fields.tx_type, fields): return Invalid
return StrictCheck(fields.tx_type, tx, fields)
\end{verbatim}
\end{quote}
\texttt{StrictCheck} は canonical TLV bytes 上での $\mathsf{TypeCheck}$、$\mathsf{c}_{id}$ 再計算、署名検証、入力消費、出力整合を含む。
排他性は \texttt{tx\_type} の一意性によって担保され、同一 $tx$ が複数の型として成立しない。
DoS 耐性の観点では、canonical TLV bytes と cheap check により検証コストを一定化し、形推測による余分な探索を避ける。
合意安全性の観点では、\texttt{tx\_type} と byte[] を合意対象にすることで曖昧な解釈差を排除できる。
将来拡張性の観点では、\texttt{proto\_version} と \texttt{payload\_bytes} により新型の追加を互換的に扱える。

\subsection{CreateContract}
Definition 9 の CreateContract に対応する。
CreateContract は $\mathsf{aux}$ の \texttt{tx\_type} を \texttt{CREATE} とし、\texttt{t\_index}(または \texttt{schema\_id})、\texttt{init\_params\_bytes}(canonical TLV)、および \texttt{cid} を含む。
$\mathsf{c}_{id}$ は
\begin{equation}
	\mathsf{c}_{id} := H(\mathsf{tx}_{id} \parallel \mathsf{pk}_{\mathrm{from}} \parallel \mathsf{t}_{index} \parallel \mathsf{init\_params\_bytes})
\end{equation}
により再計算可能であることを要求する。
ここで $\mathsf{tx}_{id}$ は CreateContract $tx$ の ID、$\mathsf{pk}_{\mathrm{from}}$ は入力 $u$ の所有者(送信元)を表す。
$\mathsf{TypeCheck}$ は
\begin{equation}
	\mathsf{TypeCheck}(\mathsf{init\_params\_bytes}, \mathsf{ArgSchema}[\mathsf{t}_{index}])=\mathsf{true}
\end{equation}
を満たすことを要求する。
\texttt{CreateTx} は $\mathsf{in}/\mathsf{out}$ が空である必要はなく、手数料支払い・Create+Lock 同梱のために $\mathsf{in}/\mathsf{out}$ を持ち得る。
$\mathsf{out}$ に $\mathsf{to}=0x0$ の焼却出力を必須とする設計は採らず、仮に marker 出力を設ける場合も非合意(表示用)に限る。
検証は (i) creator 認証(署名または入力所有の証明)、(ii) canonical TLV bytes 上の $\mathsf{TypeCheck}$、(iii) $\mathsf{c}_{id}$ 再計算一致を満たすことから成る。
ここで $\mathsf{ArgSchema}[\mathsf{t}_{index}]$ はテンプレートの Constructor 引数スキーマである。
DoS 対策として \texttt{CreateTx} に追加デポジットを必須化するのは第一選択ではなく、(i) canonical TLV bytes 化による検証コストの一定化、(ii) 手数料市場、(iii) 状態肥大を生む Lock/Execute の規律を優先する。

\subsection{LockUtxo}
Definition 9 の LockUtxo に対応する。
$\mathsf{aux}$ の \texttt{tx\_type} は \texttt{LOCK} とし、\texttt{cid} と \texttt{lock\_payload}(ロック条件・必要なら amount 等)を含む。
LockUtxo は通常 $u$ を消費し、$(\mathsf{c}_{id}, \mathsf{stateTag}=s_{\mathrm{init}})$ を伴う $u^\star$ を出力する。
ロック出力は $\mathrm{P2CONTRACT}(\mathsf{c}_{id}, \mathsf{stateTag}=s_{\mathrm{init}})$ のように $\mathsf{c}_{id}$ と状態タグに結びつく形式であり、検証規則で一致を確認する。
検証は入力の正当消費、ロック出力形式の一致、および必要なら amount の整合を含む。
CreateContract と LockUtxo は実装上同一 $tx$ に畳めるが、\texttt{tx\_type} の排他性は維持される必要がある。
本稿では (i) \texttt{tx\_type=LOCK} とし、\texttt{payload\_bytes} に create 情報(\texttt{init\_params\_bytes} 等)を含める方式を採用案とする。
将来拡張として (ii) \texttt{tx\_type=CREATE\_LOCK} の複合型を導入する方式もあり得る。

\subsection{ExecuteContract}
$u^\star$ を入力として消費する ExecuteContract $tx_\ell$ を指す。
ここで $\ell$ は遷移ラベルである。
$\mathsf{aux}$ の \texttt{tx\_type} は \texttt{EXECUTE} とし、\texttt{cid}、\texttt{call\_data\_bytes}(canonical)、\texttt{proof/witness}、\texttt{state\_ref} を含む。
この $tx_\ell$ は現在状態(または状態ロック)の $u^\star$ を $\mathsf{in}$ として消費しなければならず、$\mathsf{aux}$ だけでは二重実行を防げないことを明示する。
出力は次状態の $u^{\star\prime}$ を生成し、$(\mathsf{c}_{id}, \mathsf{stateTag}=s')$ で状態更新を表す。
admissible 判定は第4章の $\mathsf{Pred}_{\tau,\ell}$ と $\mathsf{TransSpec}_{\tau,\ell}$ に従い、(i) 消費される $\mathsf{in}$ が $(\mathsf{c}_{id}, \mathsf{stateTag}=s)$ に一致、(ii) 遷移検証(Pred/TransSpec)に合格、(iii) 出力が次状態仕様に一致、を満たす。

継続遷移であれば $\mathsf{out}(tx)$ は次状態の $u^{\star\prime}(\dots,\mathsf{c}_{id}, \mathsf{stateTag}=s')$ を含み、終端遷移であれば受益者への通常 $u$ とお釣り等のみを含む。

\subsection{FinalizeContract}
Definition 11 で導入した確定遷移 FinalizeContract に対応する。
期限経過 $\operatorname{after}(d)$ や $\operatorname{after}(t)$ に対応する遷移ラベル $\ell_{\mathrm{final}}$ を与え、admissibility と同じ枠組みで検証可能にする。
Contract Transaction 共通ヘッダに従い、\texttt{tx\_type} は \texttt{FINALIZE} とする(任意)。
誰でも起動可能な確定遷移として扱い、活性(デッドロック回避)に接続する。
