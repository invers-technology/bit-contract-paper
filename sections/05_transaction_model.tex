\section{Transaction Model}
本章では、$u$ を用いたスマートコントラクトの実現に関する $tx$ を論じる。
4章で与えたテンプレート/インスタンス/$u^\star$ および admissibility の定義を前提に、$tx$ の型(classes)として作成・実行・確定の流れを整理する。
まず、$tx$ を \textbf{Transfer Transaction} と \textbf{Contract Transaction} に二分する。
Transfer Transaction はベースレイヤの通常価値移転であり、本稿の追加検証の対象外(前提)とする。
Contract Transaction はテンプレート参照・$\mathsf{c}_{id}$・$\mathsf{stateTag}$・$\mathsf{TransSpec}$ により検証される契約系遷移である。
典型的なライフサイクルは
\[
	\texttt{Create} \rightarrow \texttt{Lock} \rightarrow \texttt{Execute}^{\ast} \rightarrow (\texttt{Finalize} / \texttt{Timeout} / \texttt{Release})
\]
であり、$\texttt{Execute}^{\ast}$ は 0 回以上の遷移を表す。
$tx$ の一般形は第4章(Formalization)の Definition 3 に従う。
以降は各“型”ごとに、Contract Transaction 共通ヘッダに基づく $\mathsf{aux}$ と $\mathsf{out}$ の追加制約のみを列挙する。
$tx$ 型の判定は $\mathsf{in}/\mathsf{out}$ の有無などの否定条件に依存せず、\texttt{tx\_type} によって一意に定める。
本章の各遷移 $\ell$ は、$\mathsf{stateTag}$ 付きの $u^\star$ を入力に取り、$\mathsf{TransSpec}$ が指定する (i) 次の $u^{\star\prime}$ への更新 または (ii) 解放($S1/S2$)の出力 を生成する Contract Transaction の遷移である。
以降の具体化では、Escrow のテンプレート $\mathsf{t}_{index}=1$ と固定する。
以下の表記は本章の $tx$ モデルに従う。
\begin{enumerate}
	\item オペレーター公開鍵 $\mathsf{pk}_C$ が、マージン率 $m$ を含めた CreateTx(CreateContract)$tx$ を送信する。
	\item 買い手公開鍵 $\mathsf{pk}_A$ が、価格 $p$ と販売者公開鍵 $\mathsf{pk}_B$ を含めた LockTx(LockUtxo)$tx$ を送信する。
	\item 2 で参照した $\mathsf{c}_{id}$ に対して ExecuteTx(ExecuteContract)$tx$($tx_\ell$)を送信し、処理を行う。
\end{enumerate}
Contract Transaction の型は \textbf{CreateTx}(CreateContract; \texttt{tx\_type=CREATE})、\textbf{LockTx}(LockUtxo; \texttt{tx\_type=LOCK})、\textbf{ExecuteTx}(ExecuteContract; \texttt{tx\_type=EXECUTE})、\textbf{FinalizeTx}(FinalizeContract; \texttt{tx\_type=FINALIZE})の4種類である。

\subsection*{Bitcoin Script 互換(Soft-fork)としての実装可能性(注記)}
\begin{enumerate}
	\item[(a)] 本稿の前提として、共有される動的状態は $\mathsf{stateTag}$ のみに限定し、その他の契約パラメータは LockTx(もしくはテンプレート参照 $\Gamma$)に固定されることを再確認する。
	\item[(b)] “既存の Bitcoin Transfer Tx だけで” Create/Lock/Execute/Finalize を厳密に表現するのは困難であり、少なくともスクリプト拘束(標準形)の追加が必要である。
	\item[(c)] 互換化の要点は次の2点である。
	      \begin{itemize}
		      \item \textbf{データ配置:} $\mathsf{c}_{id}$/$\mathsf{stateTag}$ は scriptPubKey / witness / commitment へ埋め込み可能(Taproot script-path 等)。
		      \item \textbf{遷移拘束:} TransSpec 一致(出力・次 $\mathsf{stateTag}$ 等)を合意規則で強制するには covenant / tx-introspection 系(例:OP\_CHECKTEMPLATEVERIFY 等)を要し得る。
	      \end{itemize}
\end{enumerate}
注:本稿のTx型(Create/Lock/Execute/Finalize)は、実装工数を許容すれば Taproot Script-path 等の Bitcoin Script 互換な拘束として近似できるが、遷移出力(TransSpec一致)を合意規則として厳密に強制するには covenant / tx-introspection 系の追加機構(例:OP\_CHECKTEMPLATEVERIFY 等)を要する。

\subsection{Verification Pipeline(aux / in / out をいつ見るか)}
本稿の $tx$ 検証は cheap $\rightarrow$ medium $\rightarrow$ strict の3段階とし、各段階で扱うデータ量を単調に増やす(DoS 耐性のため)。
\begin{enumerate}
	\item \textbf{Cheap prefilter(auxヘッダのみ)}
	      \begin{itemize}
		      \item \texttt{aux} の先頭には共通ヘッダ \texttt{proto\_version}, \texttt{codec\_id}, \texttt{tx\_type}, \texttt{payload\_len}, \texttt{cid}(または導出ヒント)を置く(\S5.3)。
		      \item この段階では \texttt{payload} の fully decode を行わず、ヘッダの読取と長さ整合のみを確認する。
		      \item \texttt{tx\_type} が \{\texttt{CREATE}, \texttt{LOCK}, \texttt{EXECUTE}, \texttt{FINALIZE}\} に含まれない場合は Contract Transaction として扱わず、Transfer Transaction の検証へフォールバックする。
	      \end{itemize}
	\item \textbf{Medium check(in/outの形を先に拘束)}
	      \begin{itemize}
		      \item \texttt{LOCK}/\texttt{EXECUTE}/\texttt{FINALIZE} は $\mathsf{in}/\mathsf{out}$ の拘束が本体であるため、payload の完全解釈より前に required input/output を検査する。
		      \item 具体的には、State-UTXO($u^\star$)等の required input の存在、および次 State-UTXO 等の required output の存在を型ごとに確認する(\S5.5)。
	      \end{itemize}
	\item \textbf{Strict check(payload fully decode + 検証)}
	      \begin{itemize}
		      \item canonical bytes 上で \texttt{payload} を fully decode し、型固有の TypeCheck / \texttt{cid} 再計算 / 遷移検証 / 署名・証明検証を実行する。
		      \item $tx$ 種別は \texttt{tx\_type} によって一意に定め、$\mathsf{in}/\mathsf{out}$ の欠如等の否定条件で推測してはならない。
	      \end{itemize}
\end{enumerate}
擬似コードは次である。
\begin{quote}
	\small
	\begin{verbatim}
VerifyTx(tx):
  hdr = ParseAuxHeader(tx.aux)
  if hdr.tx_type not in {CREATE,LOCK,EXECUTE,FINALIZE}: return fallback
  PrecheckInOutByType(tx, hdr.tx_type, hdr.cid)
  payload = DecodeCanonical(tx.aux, hdr)
  return StrictVerifyByType(tx, hdr, payload)
\end{verbatim}
\end{quote}

\subsection{Canonicalization and Encoding(auxはcanonical bytes)}
\begin{itemize}
	\item \texttt{aux}(および witness に載る引数・証明)は canonical bytes が合意対象であり、文字列(JSON 等)は非合意データである。
	\item Java 実装では、合意対象データの内部表現を \texttt{byte[]} として保持し、ハッシュ・署名・検証は常に \texttt{byte[]} 上で行う。
\end{itemize}

\paragraph{Canonicalization 要件.}
\begin{itemize}
	\item 整数は幅とエンディアンを固定する(例:\texttt{u8/u16/u32/u64} を big-endian)。
	\item \texttt{cid} や各種ハッシュ値は raw bytes(例:\texttt{bytes32})として扱い、16進文字列等を合意対象に含めない。
	\item \texttt{aux} 内の構造化フィールドは Tag 昇順で並べ、同一 Tag の重複を禁止する(\S5.3)。
	\item Length は固定幅(\texttt{u32})として一意に符号化する(可変長を採る場合も最短表現のみ許容し、非canonical 表現は invalid)。
	\item 文字列を合意対象に含めることは推奨しない。やむを得ず含める場合は UTF-8 に固定し、正規化規則を仕様で固定する。
\end{itemize}

\subsection{TLV(TKV) Format for aux(タグ設計・長さ制限)}
\paragraph{aux の全体構造.}
\texttt{aux} は先頭に固定長の共通ヘッダを置き、その後ろに \texttt{payload} を配置する。\texttt{payload} は TLV(Tag--Length--Value; Tag を Key と呼ぶ場合もある)の列である。

\paragraph{用語(コミットメント).}
本稿では次を用いる(以降この語を用いる):
\begin{itemize}
	\item \textbf{\texttt{staticCommit}:} 不変条件(当事者・閾値・期限・テンプレート参照 $\Gamma$・固定パラメータ等)のコミットメント(hash)。
	\item \textbf{\texttt{stateTag}:} 動的状態のコミットメント(hash)。\texttt{ExecuteTx} によってのみ更新される。
\end{itemize}

\paragraph{共通ヘッダ(固定順).}
\begin{itemize}
	\item \texttt{proto\_version}: \texttt{u8}
	\item \texttt{codec\_id}: \texttt{u8}(\texttt{0}=\texttt{TLV})
	\item \texttt{tx\_type}: \texttt{u8}(\texttt{1=CREATE}, \texttt{2=LOCK}, \texttt{3=EXECUTE}, \texttt{4=FINALIZE})
	\item \texttt{payload\_len}: \texttt{u32}(\texttt{payload} の総バイト長)
	\item \texttt{cid}: \texttt{bytes32}(CREATE では宣言値だが strict で再計算一致を要求する)
\end{itemize}
ヘッダの整数は符号なしの big-endian とし、\texttt{payload\_len} は上限 $L_{\max}$(例:4096 bytes)を超えてはならない。
また、\texttt{payload\_len} は \texttt{aux} の実バイト長と整合していなければならない(不一致は invalid)。

\paragraph{TLV ルール(payload).}
\begin{itemize}
	\item Tag: \texttt{u16}(big-endian)
	\item Length: \texttt{u32}(big-endian)
	\item Value: raw bytes
	\item 並び順: Tag 昇順(canonical)
	\item 重複 Tag: 禁止(duplicate tag は invalid)
	\item 未知 Tag: strict では reject(\texttt{proto\_version} による将来拡張で扱う)
\end{itemize}

\begin{table}[t]
	\centering
	\small
	\begin{tabular}{lllll}
		\toprule
		Tag             & フィールド名                     & 型(Value)         & 必須               & 最大長            \\
		\midrule
		\texttt{0x0100} & \texttt{TAG\_TINDEX}       & \texttt{u32}     & CREATE           & 4              \\
		\texttt{0x0101} & \texttt{TAG\_NONCE}        & \texttt{u64}     & CREATE           & 8              \\
		\texttt{0x0102} & \texttt{TAG\_INITPARAMS}   & \texttt{bytes}   & CREATE           & $\le L_{\max}$ \\
		\texttt{0x0103} & \texttt{TAG\_CREATOR}      & \texttt{bytes}   & opt              & $\le L_{\max}$ \\
		\texttt{0x0104} & \texttt{TAG\_SIG}          & \texttt{bytes}   & opt              & $\le L_{\max}$ \\
		\midrule
		\texttt{0x0200} & \texttt{TAG\_STATICCOMMIT} & \texttt{bytes32} & LOCK             & 32             \\
		\texttt{0x0201} & \texttt{TAG\_INITSTATETAG} & \texttt{bytes32} & LOCK             & 32             \\
		\texttt{0x0202} & \texttt{TAG\_STATICREVEAL} & \texttt{bytes}   & opt              & $\le L_{\max}$ \\
		\midrule
		\texttt{0x0300} & \texttt{TAG\_METHOD}       & \texttt{u32}     & EXECUTE          & 4              \\
		\texttt{0x0301} & \texttt{TAG\_ARGS}         & \texttt{bytes}   & EXECUTE          & $\le L_{\max}$ \\
		\texttt{0x0302} & \texttt{TAG\_PROOF}        & \texttt{bytes}   & EXECUTE/FINALIZE & $\le L_{\max}$ \\
		\texttt{0x0303} & \texttt{TAG\_NEXTSTATETAG} & \texttt{bytes32} & EXECUTE          & 32             \\
		\midrule
		\texttt{0x0400} & \texttt{TAG\_FINALREASON}  & \texttt{u8}      & FINALIZE         & 1              \\
		\texttt{0x0402} & \texttt{TAG\_PAYOUTCOMMIT} & \texttt{bytes32} & opt              & 32             \\
		\bottomrule
	\end{tabular}
	\caption{aux payload の TLV タグ(例)}
\end{table}

\subsection{State Model and Data Placement(stateTagのみ動的)}
\S5.3 で定義した \texttt{staticCommit} と \texttt{stateTag} を用い、State-UTXO($u^\star$)は実装上 $(\mathsf{c}_{id}, \texttt{staticCommit}, \texttt{stateTag})$ にコミットするロック出力である(script/commitment への埋め込み)。

\paragraph{単一状態(single-state).}
\begin{itemize}
	\item 1 つの契約インスタンス(\texttt{cid})につき、未消費の State-UTXO は常に高々 1 つでなければならない。
	\item \texttt{LockTx} は State-UTXO をちょうど 1 つ生成し、\texttt{ExecuteTx} は State-UTXO をちょうど 1 つ消費して次 State-UTXO をちょうど 1 つ生成する。
	\item 不変条件は \texttt{staticCommit} に固定し、遷移により更新され得る共有データは \texttt{stateTag} のみに限定する。
\end{itemize}

\paragraph{過去参照の回避.}
\texttt{aux} に不変条件の実体を直接埋め込む設計は、検証時に過去 $tx$ 参照や参照関係の追跡を必要とし得る。
本稿では、不変条件は LockTx(またはテンプレート参照 $\Gamma$)で確定し、そのコミットメント \texttt{staticCommit} を State-UTXO 側で引き回すことで、\texttt{ExecuteTx} が過去参照なしに検証できるようにする。

\subsection{Additional Constraints per Tx Type(Create/Lock/Execute/Finalize)}
本節では、\S5.1 のパイプラインに沿って、型ごとの追加制約を規範化する(A: Identification, B: Precheck, C: Strict Verify, D: Data Placement)。

\subsubsection{CreateTx(CREATE; CreateContract)}
\paragraph{A. Identification (aux header)}
\begin{itemize}
	\item ヘッダの \texttt{tx\_type} は \texttt{CREATE} でなければならない。
\end{itemize}
\paragraph{B. Precheck (in/out constraints before full decode)}
\begin{itemize}
	\item CreateTx は $\mathsf{in}/\mathsf{out}$ の有無で判定しない(手数料支払い等のために $\mathsf{in}/\mathsf{out}$ を持ち得る)。
	\item 状態肥大を避けるため、CreateTx が State-UTXO を生成しないことを推奨する(状態生成は LockTx に寄せる)。
\end{itemize}
\paragraph{C. Strict Verify (payload decode + checks)}
\begin{itemize}
	\item payload から \texttt{tindex/schema\_id}, \texttt{nonce}, \texttt{init\_params\_bytes} を取得する。
	\item $\mathsf{TypeCheck}(\texttt{init\_params\_bytes}, \mathsf{ArgSchema}[\texttt{tindex}])=\mathsf{true}$ を満たさなければならない。
	\item \texttt{creator\_id} は署名検証等により同定される送信主体である(明示する場合は \texttt{TAG\_CREATOR} に格納する)。
	\item \texttt{cid} は $\texttt{cid}=H(\texttt{creator\_id} \parallel \texttt{nonce} \parallel \texttt{tindex} \parallel \texttt{init\_params\_bytes})$ に一致しなければならない。
	\item creator 認証(署名または入力所有の証明)に合格しなければならない。
\end{itemize}
\paragraph{D. Data Placement (out/witness/aux)}
\begin{itemize}
	\item \texttt{init\_params\_bytes} は canonical bytes とする。
	\item JSON 等の文字列表現は非合意であり、保持する場合は optional debug に限定する。
	\item payload(TLV)例:\texttt{TAG\_TINDEX(u32)}, \texttt{TAG\_NONCE(u64)}, \texttt{TAG\_INITPARAMS(bytes)}, \texttt{TAG\_CREATOR(bytes or implied)}, \texttt{TAG\_SIG(bytes)}.
\end{itemize}

\subsubsection{LockTx(LOCK; LockUtxo)}
\paragraph{A. Identification (aux header)}
\begin{itemize}
	\item ヘッダの \texttt{tx\_type} は \texttt{LOCK} でなければならない。
	\item \texttt{cid} を取得し、以降の in/out 拘束に用いる。
\end{itemize}
\paragraph{B. Precheck (in/out constraints before full decode)}
\begin{itemize}
	\item outputs に State-UTXO が必ず 1 つ存在しなければならない。
	\item その State-UTXO は \texttt{cid} と \texttt{staticCommit} を含み(script/commitment への埋め込み)、初期 \texttt{stateTag} を含まなければならない。
\end{itemize}
\paragraph{C. Strict Verify (payload decode + checks)}
\begin{itemize}
	\item payload から \texttt{staticCommit} 生成に必要なパラメータ(または \texttt{staticReveal})を取得し、State-UTXO がコミットする \texttt{staticCommit} と一致することを検証する。
	\item 初期 \texttt{stateTag} の正当性(\texttt{init\_params\_bytes} 等から導出する場合は一致)を検証する。
	\item ロック条件(当事者/閾値/期限等)が \texttt{staticCommit} に含まれる(不変化される)ことを検証する。
\end{itemize}
\paragraph{D. Data Placement (out/witness/aux)}
\begin{itemize}
	\item 不変条件の実体は off-chain でもよい。オンチェーンでは \texttt{staticCommit} にコミットされていればよい。
	\item amount は \texttt{out.value} を正とし、aux に重複保持しない(齟齬防止)。
	\item payload(TLV)例:\texttt{TAG\_STATICCOMMIT(bytes32)}, \texttt{TAG\_INITSTATETAG(bytes32)}, \texttt{TAG\_STATICREVEAL(bytes)}, \texttt{TAG\_SIG(bytes)}.
\end{itemize}

\subsubsection{ExecuteTx(EXECUTE; ExecuteContract)}
\paragraph{A. Identification (aux header)}
\begin{itemize}
	\item ヘッダの \texttt{tx\_type} は \texttt{EXECUTE} でなければならない。
	\item \texttt{cid} を取得し、State-UTXO の整合に用いる。
\end{itemize}
\paragraph{B. Precheck (in/out constraints before full decode)}
\begin{itemize}
	\item inputs に「現在の State-UTXO」が必ず 1 つ含まれなければならない(単一状態)。
	\item outputs に「次の State-UTXO」が必ず 1 つ存在しなければならない。
	\item input と output で \texttt{cid} と \texttt{staticCommit} は不変でなければならない(一致)。
\end{itemize}
\paragraph{C. Strict Verify (payload decode + checks)}
\begin{itemize}
	\item payload から \texttt{method\_id}, \texttt{args\_bytes}, \texttt{proof/witness}, \texttt{next\_stateTag} を取得する。
	\item \texttt{next\_stateTag} が $\mathsf{Transition}(\texttt{input.stateTag}, \texttt{method\_id}, \texttt{args\_bytes}, \texttt{proof}, \texttt{staticCommit})$ と一致することを検証する。
	\item output がコミットする \texttt{stateTag} が \texttt{next\_stateTag} に一致しなければならない。
	\item 支払いが伴う場合は、value 更新規則(受取先/上限等が \texttt{staticCommit} で規定されること)を検証する。
\end{itemize}
\paragraph{D. Data Placement (out/witness/aux)}
\begin{itemize}
	\item call\_data/proof は witness(または aux payload)でよいが canonical bytes とする。
	\item 過去 $tx$ 参照は禁止とし、必要な不変条件は \texttt{staticCommit} と整合させて検証する。
	\item payload(TLV)例:\texttt{TAG\_METHOD(u32)}, \texttt{TAG\_ARGS(bytes)}, \texttt{TAG\_PROOF(bytes)}, \texttt{TAG\_NEXTSTATETAG(bytes32)}.
\end{itemize}

\subsubsection{FinalizeTx(FINALIZE; FinalizeContract)}
\paragraph{A. Identification (aux header)}
\begin{itemize}
	\item ヘッダの \texttt{tx\_type} は \texttt{FINALIZE} でなければならない。
\end{itemize}
\paragraph{B. Precheck (in/out constraints before full decode)}
\begin{itemize}
	\item inputs に State-UTXO を必ず含まなければならない。
	\item outputs に State-UTXO を含めない(終端)ことを仕様として固定する。
\end{itemize}
\paragraph{C. Strict Verify (payload decode + checks)}
\begin{itemize}
	\item 終端条件(期限到来/双方署名等)が \texttt{staticCommit} 規定に合致することを検証する。
	\item 支払い分配が規定通りであることを検証する。
\end{itemize}
\paragraph{D. Data Placement (out/witness/aux)}
\begin{itemize}
	\item finalize の根拠(署名/証明)は witness/aux に canonical bytes として保持する。
	\item payload(TLV)例:\texttt{TAG\_FINALREASON(u8)}, \texttt{TAG\_PROOF(bytes)}, \texttt{TAG\_PAYOUTCOMMIT(bytes32)}.
\end{itemize}

\subsection{(Optional) Create+Lock同梱}
\begin{itemize}
	\item 実装上、Create と Lock を単一 $tx$ に畳み込める。
	\item ただし \texttt{tx\_type} の排他性を崩さないため、本稿では \textbf{(i) \texttt{tx\_type=LOCK} に統一し、payload に Create 相当(\texttt{tindex/init\_params/nonce} 等)も含める}方式を採用する。
	\item \texttt{tx\_type=CREATE\_LOCK} の複合型を追加する方式は、将来拡張の選択肢としてのみ言及に留める。
\end{itemize}

\subsection{FeeReserve output}
We introduce a dedicated UTXO output, \textsf{FeeReserve}, whose value is fixed to $R(V)$ and whose spending is restricted by the \textsf{Finalize} template.
The \textsf{FeeReserve} output is funded from the operator-side fee budget, not from the principal $V$.

\paragraph{Funding constraint.}
Let $\mathrm{OpFee}(V)$ denote the operator fee budget. The construction MUST satisfy:
\[
	\mathrm{OpFee}(V) \ge R(V),
\]
so that $R(V)$ can be reserved without touching the principal.
\subsection{FinalizeTx (public finalization)}
Finalization MAY be performed by any actor (operator, buyer, seller, or a third party) to ensure liveness.
Let $\mathrm{pk}_F$ denote the public key of the finalizer (the transaction issuer).

\paragraph{Spending rule.}
\textsf{FinalizeTx} consumes the \textsf{FeeReserve} UTXO and produces:
(i) an output paying exactly $b(V)$ to $\mathrm{pk}_F$, and
(ii) an output returning the remainder to the operator payout address.

\paragraph{Fee bound.}
To bound transaction fees, the \textsf{Finalize} template enforces that the operator-return output is at least $R(V)-b(V)-F_{\max}$.
Equivalently, at most $F_{\max}$ of the \textsf{FeeReserve} value can be left as miner fee.
This guarantees that finalization can be executed with bounded fee consumption while preventing diversion of the principal $V$.
