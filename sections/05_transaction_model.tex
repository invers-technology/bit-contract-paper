\section{Transaction Model}
本章では、UTXOを用いたスマートコントラクトの実現に関するトランザクションを論じる。
4章で与えたテンプレート/インスタンス/Locked UTXO および admissibility の定義を前提に、トランザクションの型(classes)として作成・実行・確定の流れを整理する。
本章で扱うトランザクションは、Contract Creation(CreateSC)、Contract Lock(Lock)、Contract Execution($tx_\ell$)、Contract Finalize($tx_{\ell_{\mathrm{final}}}$)の4種類である。
トランザクションの一般形は第4章(Formalization)の Definition 3 に従う。以降は各“型”ごとに、Definition 3 の $\mathsf{out}/\mathsf{aux}$ に課す追加制約(marker、含めるパラメータ、$\mathsf{c}_{id}$ 導出など)のみを列挙する。
本章の各遷移 $\ell$ は、Locked UTXO(stateTag付き)を入力に取り、$\mathsf{TransSpec}$ が指定する (i) 次の Locked UTXO への更新 または (ii) 解放(S1/S2)の出力 を生成する“トランザクション型”である。

\subsection{Creation}
Definition 9 の CreateSC に対応する。CreateSC は $\mathsf{out}$ に $\mathsf{to}=0x0$ の出力(marker/burn)を含み、$\mathsf{aux}$ に $\mathsf{t}_{index}$ と $\mathsf{InitParams}$ を含む。さらに
\begin{equation}
	\mathsf{TypeCheck}(\mathsf{InitParams}, \mathsf{ArgSchema}[\mathsf{t}_{index}])=\mathsf{true}
\end{equation}
を満たし、$\mathsf{c}_{id}$ は
\begin{equation}
	\mathsf{c}_{id} := H(\mathsf{deployerAddr} \parallel \mathsf{t}_{index} \parallel \mathsf{InitParams})
\end{equation}
により決定的に導出される。ここで $\mathsf{ArgSchema}[\mathsf{t}_{index}]$ はテンプレートの Constructor 引数スキーマである。

\subsection{Lock}
Definition 9 の Lock に対応する。通常 UTXO $u$ を消費し、Locked UTXO $u^\star$ を出力する。
出力には $(\mathsf{c}_{id}, \mathsf{stateTag}=s_{\mathrm{init}})$ が付与される。
CreateSC と Lock は実装上同一Txに畳めるが、概念上は分離して記述する。

\subsection{Execution}
Locked UTXO $u^\star$ を入力として消費するトランザクション $tx_\ell$($\ell$ は遷移ラベル)を指す。
admissible 判定は第4章の $\mathsf{Pred}_{\tau,\ell}$ と $\mathsf{TransSpec}_{\tau,\ell}$ に従う。

継続遷移であれば $\mathsf{out}(tx)$ は次状態の $u^{\star\prime}(\dots,\mathsf{c}_{id}, \mathsf{stateTag}=s')$ を含み、終端遷移であれば受益者への通常 UTXO(およびお釣り等)のみを含む。

\subsection{Finalize}
Definition 11 で導入した確定遷移に対応する。
期限経過 $\operatorname{after}(d)$ や $\operatorname{after}(t)$ に対応する遷移ラベル $\ell_{\mathrm{final}}$ を与え、admissibility と同じ枠組みで検証可能にする。
