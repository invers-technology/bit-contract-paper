\section{Problem and Requirements}

\subsection{Problem}
本章では、身元が不確かな環境で、プラットフォームが決済補助(エスクロー)を提供する市場に内在する制度リスクとして、custodial escrow failure / marketplace disappearance risk を問題化する。
本稿の関心は特定の違法取引領域にあるのではなく、当事者が実世界の身元・所在地を相互に確証しにくい環境で市場取引を反復可能にするために導入される「信頼の代替装置」が、同時に単一障害点となり得るという構造にある。

この構造の中心にあるのが、資金の一時保有(custody)である。
エスクロー等の決済補助がプラットフォーム運営に集中するほど、運営は資金と手続を握る度合いを増し、利用者資金の不正流用(custodial misappropriation)や、サイトの消失・閉鎖による資金回収不能(marketplace disappearance)といった失敗モードが顕在化する。
俗称として「Exit scam」と呼ばれる事象は、この失敗モードの代表例であり、運営が出金を停止して預託資金を持ち逃げすることによって発生する。

計測研究は、高リスク環境においてこの種の事象が参加者に大きな経済的損失をもたらし、市場の分断や参加者の移行(migration)を通じて不安定性のサイクルが形成され得ることを報告している。
本稿は、そうした領域を「設計対象」として前面化するのではなく、観測研究が厚い“代表例”として参照し、一般化可能な制度リスクとしてモデル化する(付録Aのエビデンス・マトリクス参照)。

\subsection{Model}
本稿では、匿名市場(darknet market)を主語に置かず、より中立に identity-opaque marketplace with custodial escrow(IOM-CE) を対象モデルとして定義する。
IOM-CEとは、当事者が実世界の身元・所在地を相互に確証しにくい環境で、オンライン上の取引インフラが提供され、決済補助として資金が第三者(典型的にはプラットフォーム運営)に一時保有され得る市場である。

DNM等はこの定義の「極端例」に該当し、計測研究により制度リスクの観測が相対的に厚い点で参照価値があるが、本稿の目的は特定領域の運用手順を示すことではない。
以降のモデル化は、資金拘束と解放条件の設計という抽象レベルに留め、具体的な運営方法・回避手段・実務手順に踏み込まない。

\begin{itemize}
	\item Buyer/Seller:需要側・供給側の機能ロール
	\item Operator/Custodian:取引インフラ提供者であり、資金の一時保有を担い得る主体で、単一障害点になり得る
	\item Escrow / Payment facilitation:価値移転の補助として資金拘束を行う仕組みであり、エスクロー等を含む
	\item Reputation / Feedback:身元不確証環境で信頼を代替する情報制度であり、補助制度として位置づける
	\item Disappearance / Scams / Breakdown:市場の消失・閉鎖・詐欺的イベント等の制度リスク
	\item Resilience / Migration:閉鎖等を受けた移行・エコシステムの存続形態であり、移動による持続等を含む
\end{itemize}

\subsubsection{Actors and Roles}

Escrow/持ち逃げの問題に限定するため、アクターは次の最小3者で十分である。

\begin{itemize}
	\item Buyer:買い手として資金を預託し、条件が満たされれば支払を完了したい主体
	\item Seller:売り手として条件が満たされれば資金の解放を受けたい主体
	\item Operator:運営/custodianとして決済補助の資金を一時保有し得る主体であり、単一支配が成立し得る点がリスクの核心
\end{itemize}

評判(reputation)は、信頼形成の補助制度として重要であるが、IOM-CEにおける中心的失敗モード(custodial misappropriation)を規定する最小モデルの必須要素ではないため、本稿では「既存の緩和策」として後述する。

\subsubsection{Workflow}
IOM-CEを、資金の状態遷移として最小限に表現する。

\begin{itemize}
	\item $S0$: Funded escrow:Buyerが資金をエスクローとして拘束
	\item $S1$: Release to Seller:資金がSellerへ解放
	\item $S2$: Refund to Buyer:資金がBuyerへ返還
\end{itemize}
このとき問題となる失敗遷移は次である。

$F$: Misappropriation:Operatorが資金を自己に移転(または回収不能化)する遷移

本稿で扱う中心問題は、$S0$から$S1/S2$への正当遷移を「人間の裁量」ではなく、検証可能な条件として固定しつつ、$F$(Misappropriation)を制度として排除・拘束することである。
言い換えれば、IOM-CEの核心は「取引の全過程」を一般化することではなく、資金拘束の単一障害点(custody point)を同定し、その構造的リスクを縮退させることである。

\subsection{Existing solutions and gaps}
IOM-CEにおけるリスクに対して、既存システムは複数の緩和策を採用してきたが、それらは万能ではない。

D1. Reputation / Feedback:既存解

身元確認が困難な環境では、参加者は仮名的な評判・レビューに依存し、信頼を代替する。
しかし評判は偽レビュー等で操作され得て、信頼問題を完全には解消しない。
さらに、評判は「資金が運営の単独支配下にある」という構造そのものを除去しないため、custodial misappropriationに対する根本対策にはならない。

D2. Multi-signature escrow:既存解

単独管理を避けるため、マルチシグ等で資金アクセスを分散する試みがある。
これはOperator単独による$F$(Misappropriation)を抑止し得る一方で、鍵管理・参加者の利害・手続の透明性といった運用・ガバナンスが残存する。
したがって、緩和策は「単独持ち逃げの難化」には寄与しても、監査可能性や手続の決定性を上限付きで保証する枠組みにはなりにくい。

\subsection{Issues}
本稿が扱う中心問題はエスクローそれ自体ではなく、IOM-CEで要求される資金拘束・解放のロジックを、汎用VMの任意計算に頼らず、監査可能性が上限付きのテンプレートとして提供できるかである。
既存のVMベース方式では「書ける/書けない」以前に、次の課題が残り得る。

監査負荷が上限化できない:任意計算はコード規模と依存が膨張しやすく、利用者がトランザクションに署名する時点で取引意味(資金拘束・解放条件)を点検できず、監査やフロントエンド等の第三者に依存しやすい
アップグレードや運用裁量が残る:間接参照や更新可能設計が、実質的なトラスト依存を増大させ得る
カストディ構造を制度として縮退しにくい:資金拘束と解放条件が「実装+運用」に跨り、単一支配の残余が残り得る

対象を、資金拘束と解放条件の決定性、およびカストディ集中に起因する失敗遷移$F$の制度的排除/拘束に限定する。

\subsection{Requirements}
上記の問題設定から、本稿は次の要件を導出する。

R1: Constrained custody / Non-custodial settlement
Operatorが単独で資金を不正流用できる遷移$F$を、制度(検証規則)として排除または強く拘束すること。

R2: Deterministic admissibility:許容条件の決定性
$S0 \to S1$(支払)および$S0 \to S2$(返金)の許容条件を、外部の裁量ではなく検証可能な条件として表現できること。

R3: Bounded auditability:監査可能性の上限
契約ロジックはテンプレートとして制約され、利用者・監査者が点検できる複雑性の上限を持つこと。
テンプレートは作成後にアップグレードできない。

R4: Local state / minimal shared state
グローバルステートへの依存を避け、(e)UTXO等の局所状態とロック指向の資産移転で表現できること。
共有状態は $\mathsf{stateTag}$ のみに限定する。

\subsection{Research Questions}
\paragraph{RQ1:決定的表現可能性}
VMを用いず、形式化可能なテンプレート集合として、IOM-CEにおける $\mathrm{deposit} / \mathrm{release} / \mathrm{refund}$ を決定的に表現できるか?

\paragraph{命題 P1:決定的表現可能性}
焦点ケース(IOM-CE)の状態機械($S0, S0D, S0X, S1, S2$)と各遷移ラベル $\ell$ について、テンプレート $\tau$ とインスタンス $\mathsf{c}_{id}$ を固定すれば、任意の $\Sigma$ と $tx$ に対して $\mathsf{Pred}_{\tau,\ell}(\Sigma, tx, \mathsf{c}_{id})$ の真偽が一意に定まり、真である場合に限り $\mathsf{out}(tx) \in \mathsf{TransSpec}_{\tau,\ell}(\Sigma, \mathsf{c}_{id})$ が成立する。

\paragraph{RQ2:運用裁量の縮退}
そのとき、運用裁量をどこまで縮退/移送できるか?

\paragraph{定義(運用裁量の残余)}
ある役割 $r$ について、「$r$ の単独署名(または単独操作)だけで到達できる資金移動遷移の集合」を $\mathrm{Disc}(r)$ と呼ぶ。

\paragraph{命題 P2:Operator単独裁量の不在}
$\mathrm{Disc}(\mathsf{Operator})=\varnothing$。
すなわち、運営 $C$ の単独署名だけで $S1/S2$ に到達する admissible な遷移は存在しない。

\paragraph{命題 P2':裁量の移送}
運営が関与する仲裁遷移は存在するが、それらは常に $(\mathsf{pk}_A \text{ or } \mathsf{pk}_B)$ の署名を伴うため、裁量は「運営単独」から「当事者の同意の証跡」へ移送される。

\paragraph{RQ3:不変条件の保証範囲}
資金保全・権限安全・活性(デッドロックなし)を、不変条件としてどう定義し、どの範囲まで保証できるか?

\paragraph{定理 T1:Safety}
任意の admissible $tx$ は価値保存と受益者制約を破らない($\mathsf{TransSpec}$ により拘束)。

\paragraph{定理 T2:Authority Safety}
運営 $C$ 単独の不正流用遷移 $F$ は admissible にならない。

\paragraph{定理 T3:Liveness}
期限条件($\operatorname{after}(d), \operatorname{after}(t)$)により、ロック状態が永久残留しない(FinalizeContract/Timeout が誰でも起動可能)。

本章で用いたIOM-CEの構成要素(Buyer/Seller/Operator、Escrow、消失・移行等)が先行研究で反復的に観察されることは、付録Aのエビデンス・マトリクスで可視化する。
高リスク環境に関する計測研究は本稿の問題設定に経験的根拠を与えるが、本稿はそれらを設計対象としてではなく、抽象モデル化の根拠として引用する。
