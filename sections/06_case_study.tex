\section{Case Study: Escrow Misappropriation Resistance}
本ケーススタディは匿名市場に関する先行研究を組み合わせて構成している(\cite{Christin2013,Soska2015,Tzanetakis2016,Spagnoletti2022})。

\subsection{Variables}
\begin{table}[htbp]
	\centering
	\small
	\setlength{\tabcolsep}{6pt}
	\renewcommand{\arraystretch}{1.1}
	\begin{tabular}{ll}
		\toprule
		Buyerアドレス & $\Buyer$ \\
		Sellerアドレス & $\Seller$ \\
		Operatorアドレス & $\Oper$ \\
		商品価格 & $p$ \\
		マージン率 & $m$ \\
		配送期間 & $d$ \\
		コントラクトタイムアウト($t\ge d$) & $t$ \\
		Finalize報酬率($0 \le b < 1$) & $b$ \\
		\bottomrule
	\end{tabular}
\end{table}

\subsection{Template State Transition}
$\Szero$: Funded escrow(資金拘束):Buyerが資金をエスクローとして拘束
$\SzD$: Delivered pending(配送通知済み・検収猶予中):Sellerが発送の完了を通知
$\SzX$: Disputed(紛争中:タイムアウトor仲裁のみ):Buyerが発送の完了を認めず紛争
$\Sone$: Release to Seller(支払):資金がSellerへ解放
$\Stwo$: Refund to Buyer(返金):資金がBuyerへ返還

遷移の全体像は次のように要約できる。
\begin{itemize}
	\item $\BuyerLock$ で資金を拘束し $\Szero$ を開始。
	\item $\Szero \to \SzD$($\SellerDeliver$)、$\SzD \to \Sone$($\BuyerConfirm$ または $\after(d)$)。
	\item $\SzD \to \SzX$($\BuyerDispute$)、$\SzX \to \Sone/\Stwo$(仲裁)。
	\item 取消およびタイムアウトにより $\Szero/\SzD/\SzX \to \Stwo$ が可能。
\end{itemize}

\begin{transitions}
	\tr{\Szero}{p\,\uA}{\to p\,\Lu{\Szero}}
	\tr{\SzD}{p\,\Lu{\Szero}}{\to p\,\Lu{\SzD}}
	\tr{\SzX}{p\,\Lu{\SzD}}{\to p\,\Lu{\SzX}}
	\tr{\Sone}{p\,\LUTXO(\cid,\stag\in\{\SzD,\SzX\})}{\to p(1-m-b)\,\uB\ \text{and}\ \pfee{p}{m}\,\uC\ \text{and}\ \pfee{p}{b}\,\UTXO_{X}}
	\tr{\Stwo}{p\,\LUTXO(\cid,\stag\in\{\Szero,\SzD,\SzX\})}{\to p\,\uA}
\end{transitions}

\subsection{Template InitParams}
$\mathsf{InitParams}=(m, d, t, b)$
マージン率: $m$
配送期間:$d$
コントラクトタイムアウト: $t$
Finalize報酬率: $b$
$b$ はテンプレート固定(CreateSC で確定)であり、$0 \le b \le 1-m$ を仮定する。

\subsection{Transition Labels}
Operatorが $\mathsf{InitParams}$ の設定と CreateSC の送信を行い、$\mathsf{deployerAddr}$ として $\cid$ が導出される
$d$:SellerConfirm後の“検収猶予”
$t$:全体の最終期限($t \ge d$ を仮定)

$\BuyerLock(\sigA, \Seller, p)$
Buyerが資産をLockし、$\Szero$が開始

$\SellerDeliver(\sigB)$
$\Szero$からSellerの署名で $\stag=\SzD$ に移行し、検収猶予 $d$ のカウントを開始する

$\BuyerConfirm(\sigA)$
Buyerの署名で$\SzD \to \Sone$

$\BuyerDispute(\sigA)$
$\before(d)$ のみ有効
Buyerの署名で異議申し立てで$\SzD \to \SzX$

$\BuyerCancel(\sigA)$
$\Szero$または$\SzD$からのみ
Buyerの署名でSellerのキャンセル待ち

$\SellerAcceptCancel(\sigB)$
$\Szero$または$\SzD$からBuyerCancel後、Sellerの署名で$\Stwo$に移行

$\OpRel(\sigC + (\sigA \text{ or } \sigB))$
Operator仲裁で$\Sone$の状態に移行

$\OpRef(\sigC + (\sigA \text{ or } \sigB))$
Operator仲裁で$\Stwo$の状態に移行

$\Finalize()$
誰でも起動可能
実行者は受取先アドレス $X$ を出力に含めて bounty を受け取る($\mathsf{Pred}$ は $\mathsf{out}(tx)$ の $X$ を参照、署名不要)。
$\SellerDeliver$ の後、$d$時間が経過し、$\stag=\SzD$ のままなら$\SzD \to \Sone$

$\Timeout()$
誰でも起動可能
$t$の経過で$\Timeout$ 遷移が起動し、Buyerに資金移動し、$\Stwo$

遷移ラベルごとの $\mathsf{Pred}_{\tau,\ell}$ と $\mathsf{TransSpec}_{\tau,\ell}$ を表に整理する。表中の不変条件は S=Safety, A=Authority Safety, L=Liveness を表す。取消フラグは Locked UTXO に付随する補助情報として扱い、$(\cid, \stag)$ と同様に script / aux / commitment / metadata に埋め込まれ得る。
\begin{table}[htbp]
	\centering
	\footnotesize
	\setlength{\tabcolsep}{3pt}
	\renewcommand{\arraystretch}{1.15}
	\begin{tabular}{p{0.14\textwidth}p{0.08\textwidth}p{0.19\textwidth}p{0.23\textwidth}p{0.12\textwidth}p{0.12\textwidth}}
		\toprule
		遷移ラベル $\ell$           & 前提状態         & $\mathsf{Pred}_{\tau,\ell}$ が見る要素                            & $\mathsf{TransSpec}_{\tau,\ell}$(出力仕様)            & ブロードキャスト           & 不変条件    \\
		\midrule
		$\BuyerLock$          & --           & $\sigA$, $p$                                       & $p$ を拘束し、$\Lu{\Szero}$ を生成                & $\Buyer$                & S       \\
		$\SellerDeliver$      & $\Szero$     & $\sigB$                                             & $\stag=\SzD$ に更新(ロック継続)                & $\Seller$               & S, L    \\
		$\BuyerConfirm$       & $\SzD$       & $\sigA$                                             & $\pmarg{p}{m}$ を $\Seller$、$\pfee{p}{m}$ を $\Oper$ へ        & $\Buyer$                & S, A, L \\
		$\BuyerDispute$       & $\SzD$       & $\sigA$, $\before(d)$                              & $\stag=\SzX$ に更新(ロック継続)                & $\Buyer$                & A, L    \\
		$\BuyerCancel$        & $\Szero/\SzD$ & $\sigA$                                             & 取消フラグを付与(ロック継続)                       & $\Buyer$                & A       \\
		$\SellerAcceptCancel$ & $\Szero/\SzD$ & $\sigB$, 取消フラグ                                  & $p$ を $\Buyer$ に返金($\Stwo$)                   & $\Seller$               & S, A, L \\
		$\OpRel$              & $\SzX$       & $\sigC$ と $(\sigA \text{ or } \sigB)$             & $\pmarg{p}{m}$ を $\Seller$、$\pfee{p}{m}$ を $\Oper$ へ        & $\Oper + (\Buyer/\Seller)$ & S, A, L \\
		$\OpRef$              & $\SzX$       & $\sigC$ と $(\sigA \text{ or } \sigB)$             & $p$ を $\Buyer$ に返金($\Stwo$)                   & $\Oper + (\Buyer/\Seller)$ & S, A, L \\
		$\Finalize$           & $\SzD$       & $\after(d)$, 受取先 $X$                              & $p(1-m-b)$ を $\Seller$、$\pfee{p}{m}$ を $\Oper$、$\pfee{p}{b}$ を $X$ へ & Anyone                   & S, L    \\
		$\Timeout$            & $\Szero/\SzD/\SzX$ & $\after(t)$                                   & $p$ を $\Buyer$ に返金($\Stwo$)                   & Anyone                   & S, L    \\
		\bottomrule
	\end{tabular}
\end{table}

\subsection{Property}
Safety(運営の持ち逃げ抑止)
$\Oper$は単独で$\Sone/\Stwo$を実行できず、資金移動には $(\Oper + (\Buyer \text{ or } \Seller))$ が必要(または自動遷移)。
⇒ $\Oper$単独のmisappropriation($F$)を制度的に排除。

Liveness(決着性)
$\SzD$は$\BuyerConfirm$か$\after(d)$の$\Finalize$で$\Sone$へ、
$\SzX$は仲裁($\Oper + (\Buyer \text{ or } \Seller)$)か、最終的に$\after(t)$の$\Timeout$で$\Stwo$へ。
また、$\Finalize$ は誰でも実行可能で bounty があるため、期限到来後に実行されやすい。
⇒ 資金が永久拘束されない。
