\section{Case Study: Escrow Misappropriation Resistance}
本ケーススタディは匿名市場に関する先行研究を組み合わせて構成している(\cite{Christin2013,Soska2015,Tzanetakis2016,Spagnoletti2022})。

\subsection{Variables}
Buyerアドレス: $A$
Sellerアドレス: $B$
Operatorアドレス: $C$
商品価格: $p$
マージン率: $m$
配送期間: $d$
コントラクトタイムアウト: $t$($t \ge d$)

\subsection{Template State Transition}
S0: Funded escrow(資金拘束):Buyerが資金をエスクローとして拘束
S0D: Delivered pending(配送通知済み・検収猶予中):Sellerが発送の完了を通知
S0X: Disputed(紛争中:タイムアウトor仲裁のみ):Buyerが発送の完了を認めず紛争
S1: Release to Seller(支払):資金がSellerへ解放
S2: Refund to Buyer(返金):資金がBuyerへ返還

遷移の全体像は次のように要約できる。
\begin{itemize}
	\item $\mathrm{BuyerLock}$ で資金を拘束し $S0$ を開始。
	\item $S0 \to S0D$($\mathrm{SellerDeliver}$)、$S0D \to S1$($\mathrm{BuyerConfirm}$ または $\operatorname{after}(d)$)。
	\item $S0D \to S0X$($\mathrm{BuyerDispute}$)、$S0X \to S1/S2$(仲裁)。
	\item 取消およびタイムアウトにより $S0/S0D/S0X \to S2$ が可能。
\end{itemize}

S0: $p\,\mathrm{UTXO}_A \to p\,\mathrm{UTXO}^\star(\mathsf{c}_{id}, \mathsf{stateTag}=S0)$
S0D: $p\,\mathrm{UTXO}^\star(\mathsf{c}_{id}, \mathsf{stateTag}=S0) \to p\,\mathrm{UTXO}^\star(\mathsf{c}_{id}, \mathsf{stateTag}=S0D)$
S0X: $p\,\mathrm{UTXO}^\star(\mathsf{c}_{id}, \mathsf{stateTag}=S0D) \to p\,\mathrm{UTXO}^\star(\mathsf{c}_{id}, \mathsf{stateTag}=S0X)$
S1: $p\,\mathrm{UTXO}^\star(\mathsf{c}_{id}, \mathsf{stateTag}\in\{S0D,S0X\}) \to p(1 - m)\,\mathrm{UTXO}_B \text{ and } p \cdot m\,\mathrm{UTXO}_C$
S2: $p\,\mathrm{UTXO}^\star(\mathsf{c}_{id}, \mathsf{stateTag}\in\{S0,S0D,S0X\}) \to p\,\mathrm{UTXO}_A$

\subsection{Template Constructor Arguments}
マージン率: $m$
配送期間:$d$
コントラクトタイムアウト: $t$

\subsection{Template Constructor Arguments}
$\mathrm{Constructor}(m, d, t)$
Operatorが変数の初期化と CreateSC の送信を行い、$\mathsf{deployerAddr}$ として $\mathsf{c}_{id}$ が導出される
$d$:SellerConfirm後の“検収猶予”
$t$:全体の最終期限($t \ge d$ を仮定)

$\mathrm{BuyerLock}(\mathrm{sig}(A), B, p)$
Buyerが資産をLockし、$S0$が開始

$\mathrm{SellerDeliver}(\mathrm{sig}(B))$
$S0$からSellerの署名で $\mathsf{stateTag}=S0D$ に移行し、検収猶予 $d$ のカウントを開始する

$\mathrm{BuyerConfirm}(\mathrm{sig}(A))$
Buyerの署名で$S0D \to S1$

$\mathrm{BuyerDispute}(\mathrm{sig}(A))$
$\operatorname{before}(d)$ のみ有効
Buyerの署名で異議申し立てで$S0D \to S0X$

$\mathrm{BuyerCancel}(\mathrm{sig}(A))$
$S0$または$S0D$からのみ
Buyerの署名でSellerのキャンセル待ち

$\mathrm{SellerAcceptCancel}(\mathrm{sig}(B))$
$S0$または$S0D$からBuyerCancel後、Sellerの署名で$S2$に移行

$\mathrm{OperatorMediateRelease}(\mathrm{sig}(C) + (\mathrm{sig}(A) \text{ or } \mathrm{sig}(B)))$
Operator仲裁で$S1$の状態に移行

$\mathrm{OperatorMediateRefund}(\mathrm{sig}(C) + (\mathrm{sig}(A) \text{ or } \mathrm{sig}(B)))$
Operator仲裁で$S2$の状態に移行

$\mathrm{ContractFinalize}()$
誰でも起動可能
$\mathrm{SellerDeliver}$の後、$d$時間が経過し、$\mathsf{stateTag}=S0D$ のままなら$S0D \to S1$

$\mathrm{ContractTimeout}()$
誰でも起動可能
$t$の経過でFinal TransitionのBuyerに資金移動し、$S2$

遷移ラベルごとの $\mathsf{Pred}_{\tau,\ell}$ と $\mathsf{TransSpec}_{\tau,\ell}$ を表に整理する。表中の不変条件は S=Safety, A=Authority Safety, L=Liveness を表す。取消フラグは Locked UTXO に付随する補助情報として扱い、$(\mathsf{c}_{id}, \mathsf{stateTag})$ と同様に script / aux / commitment / metadata に埋め込まれ得る。
\begin{table}[htbp]
	\centering
	\footnotesize
	\setlength{\tabcolsep}{3pt}
	\renewcommand{\arraystretch}{1.15}
	\begin{tabular}{p{0.14\textwidth}p{0.08\textwidth}p{0.19\textwidth}p{0.23\textwidth}p{0.12\textwidth}p{0.12\textwidth}}
		\toprule
		遷移ラベル $\ell$           & 前提状態         & $\mathsf{Pred}_{\tau,\ell}$ が見る要素                            & $\mathsf{TransSpec}_{\tau,\ell}$(出力仕様)         & ブロードキャスト       & 不変条件    \\
		\midrule
		BuyerLock              & --           & $\mathrm{sig}(A)$, $p$                                       & $p$ を拘束し、$\mathsf{stateTag}=S0$ の Locked UTXO を生成 & Buyer          & S       \\
		SellerDeliver          & $S0$         & $\mathrm{sig}(B)$                                            & $\mathsf{stateTag}=S0D$ に更新(ロック継続)                & Seller         & S, L    \\
		BuyerConfirm           & $S0D$        & $\mathrm{sig}(A)$                                            & $p(1-m)$ を $B$、$p \cdot m$ を $C$ へ             & Buyer          & S, A, L \\
		BuyerDispute           & $S0D$        & $\mathrm{sig}(A)$, $\operatorname{before}(d)$                & $\mathsf{stateTag}=S0X$ に更新(ロック継続)                & Buyer          & A, L    \\
		BuyerCancel            & $S0/S0D$     & $\mathrm{sig}(A)$                                            & 取消フラグを付与(ロック継続)                                & Buyer          & A       \\
		SellerAcceptCancel     & $S0/S0D$     & $\mathrm{sig}(B)$, 取消フラグ                                     & $p$ を $A$ に返金($S2$)                            & Seller         & S, A, L \\
		OperatorMediateRelease & $S0X$        & $\mathrm{sig}(C)$ と $(\mathrm{sig}(A)$ or $\mathrm{sig}(B))$ & $p(1-m)$ を $B$、$p \cdot m$ を $C$ へ             & Operator + A/B & S, A, L \\
		OperatorMediateRefund  & $S0X$        & $\mathrm{sig}(C)$ と $(\mathrm{sig}(A)$ or $\mathrm{sig}(B))$ & $p$ を $A$ に返金($S2$)                            & Operator + A/B & S, A, L \\
		ContractFinalize       & $S0D$        & $\operatorname{after}(d)$                                    & $p(1-m)$ を $B$、$p \cdot m$ を $C$ へ             & Anyone         & S, L    \\
		ContractTimeout        & $S0/S0D/S0X$ & $\operatorname{after}(t)$                                    & $p$ を $A$ に返金($S2$)                            & Anyone         & S, L    \\
		\bottomrule
	\end{tabular}
\end{table}

\subsection{Property}
Safety(運営の持ち逃げ抑止)
$C$は単独で$S1/S2$を実行できず、資金移動には $(C + (A \text{ or } B))$ が必要(または自動遷移)。
⇒ $C$単独のmisappropriation($F$)を制度的に排除。

Liveness(決着性)
$S0D$は$\mathrm{BuyerConfirm}$か$\operatorname{after}(d)$の$\mathrm{Finalize}$で$S1$へ、
$S0X$は仲裁($C + (A \text{ or } B)$)か、最終的に$\operatorname{after}(t)$の$\mathrm{Timeout}$で$S2$へ。
⇒ 資金が永久拘束されない。
