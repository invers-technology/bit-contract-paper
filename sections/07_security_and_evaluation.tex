\section{Security and Evaluation}

\subsection{Threat Model and Trust Assumption}

\begin{itemize}
	\item 誰が悪い行為者になり得るか(Buyer/Seller/Operator、共謀の範囲)
	\item 何ができるか(署名偽造は不可、検閲・遅延・不作為は可能、など)
	\item チェーン側前提(タイムロック可否、最終性、手数料市場の揺れの扱い)
\end{itemize}

\subsection{Security Property}

\begin{itemize}
	\item Claim/Theorem(何が保証されるか野太く1行)
	\item Proof sketch(テンプレート検証規則/遷移表からなぜ排除されるか)
\end{itemize}

\begin{itemize}
	\item 横領が 何を意味するか(禁止したい遷移)
	\item それが なぜテンプレート検証で排除されるか(検証条件)
	\item それでも残る 攻撃面(検閲、遅延、共謀、外部決済等)
\end{itemize}

\subsection{Implementation and Evaluation}

\begin{itemize}
	\item 実装対象と再現性:どのチェーン/VM/ライブラリで、どのコミットで再現できるか(生成器・検証器・テスト手順)
	\item コスト評価:各遷移トランザクションのサイズ/手数料(感度分析も)
	\item 比較:同等機能の一般的スマコン(EVM等)に対して、監査対象・状態依存・実行失敗モードがどう減るか(表で)
\end{itemize}

評価で出す数値(サイズ、手数料、計算量、必要署名数、オンチェーン状態量)が、
5章で定義したトランザクション構造
6章のトランザクション列
に対して算出された、と追跡できる必要があります。
