\subsection{Threat Model and Security Analysis}

\paragraph{Threat model.}
Buyer/Seller/Operator のいずれも悪意主体になり得るものとし、共謀も考慮する。
暗号学的前提として署名偽造は不可能とする一方、検閲・遅延・不作為(ブロードキャストしない等)は起こり得る。
チェーン側前提としてタイムロックが利用可能であること、最終性が得られることを仮定し、手数料市場の揺れは評価時の感度として扱う。

\paragraph{What we prevent.}
本稿の焦点は、資金の不正流用(misappropriation)、単独当事者による不当な終端決着(unilateral settlement)、およびロック資金の永久凍結(freeze)である。
これらは、テンプレート検証規則と遷移表により「禁止したい遷移」を admissible から排除し、許される遷移のみを列挙することで抑止する。

\paragraph{What remains.}
一方で、検閲やネットワーク遅延、当事者間の共謀、オフチェーンの詐欺(配送・外部決済など)といった攻撃面は残り得る。

\subsection{Auditability, Cost, and Comparison}

\paragraph{Audit surface.}
監査対象は、テンプレートとして制約された $\mathsf{Pred}$ と $\mathsf{TransSpec}$、および遷移表により上限付きに限定される。
この設計は、当事者が署名時点で「何が起き得るか」を点検しやすくし、実装・運用裁量に依存する領域を縮小する。

\paragraph{Cost.}
実装対象と再現性(どのチェーン/ライブラリで、どのコミットで再現できるか)を明示し、生成器・検証器・テスト手順により追跡可能にする。
各遷移トランザクションのサイズ/手数料、オンチェーンデータ量、必要署名数などを見積もり、手数料市場の揺れに対する感度も評価する。
評価で出す数値(サイズ、手数料、計算量、必要署名数、オンチェーン状態量)が、第5章で定義したトランザクション構造と第6章のトランザクション列に対して算出された、と追跡できる必要がある。

\paragraph{Comparison.}
同等機能を一般的なVM型スマートコントラクト(例:EVM)で実現する場合と比較し、監査対象・状態依存・実行失敗モードがどのように増えやすいか、またアップグレード裁量やバグ依存の論点がどの程度残るかを整理する。
