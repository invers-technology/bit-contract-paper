\section{IOM-CE Evidence Matrix}

\subsection{IOM-CEの位置づけ}
本稿の第3章では、identity-opaque marketplace with custodial escrow(IOM-CE)という抽象モデルを用い、資金の一時保有(custody)に起因する custodial misappropriation / marketplace disappearance risk を問題化した。
IOM-CEは特定領域の運用手順を記述するためではなく、身元が確証しにくい環境で市場取引を反復可能にする制度要素を、先行研究の反復観察に基づき最小共通集合として抽出するための分析単位である。

本付録は、主要先行研究において、IOM-CEの構成要素がどの程度明示されているかを整理し、本文のモデル化が恣意的でないことを示す。
なお、対象文献には高リスク環境に関する計測研究が含まれるが、本稿はそれらを設計対象として推奨・再現する意図を持たず、抽象モデルの経験的根拠としてのみ参照する。

\subsection{構成要素の出現状況}
本節では、IOM-CE構成要素の出現状況を先行研究ごとに整理する。

\begin{table}[htbp]
	\centering
	\footnotesize
	\setlength{\tabcolsep}{4pt}
	\begin{tabular}{p{0.36\textwidth}cccc}
		\toprule
		要素(IOM-CE)                               & \shortstack{Christin                                        \\(2013)} & \shortstack{Soska\\\& Christin\\(2015)} & \shortstack{Tzanetakis\\et al.\\(2016)} & \shortstack{Spagnoletti\\et al.\\(2022)} \\
		\midrule
		Buyer/Seller:需要・供給ロール                    & \checkmark           & \checkmark & \checkmark & \checkmark \\
		Operator/Custodian:運営/資金保有主体             & \checkmark           & \checkmark & \checkmark & \checkmark \\
		Escrow / Payment facilitation:エスクロー等     & \checkmark           & \checkmark & \checkmark & \checkmark \\
		Reputation / Feedback:評判・レビュー            & \checkmark           & \checkmark & \checkmark & \checkmark \\
		Disappearance / Scams / Breakdown:消失・詐欺等 & \checkmark           & \checkmark & \checkmark & \checkmark \\
		Resilience / Migration:移行・存続形態           & \checkmark           & \checkmark & 必要に応じて     & \checkmark \\
		\bottomrule
	\end{tabular}
\end{table}

\subsection{TLV Canonical Encoding:Test Vectors}
本節は、Contract Transaction の TLV(Type-Length-Value)規範仕様に対する具体例を示す。
Type 割当はプロトコル定数とし、以下の最小例を用いる。
\begin{itemize}
	\item \texttt{0x0001}:\texttt{proto\_version}(u16)
	\item \texttt{0x0002}:\texttt{tx\_type}(u8;01=CREATE, 02=LOCK, 03=EXECUTE, 04=FINALIZE)
	\item \texttt{0x0003}:\texttt{cid}(32 bytes)
	\item \texttt{0x0004}:\texttt{t\_index}(u32)
	\item \texttt{0x0005}:\texttt{l\_index}(u32)
	\item \texttt{0x0006}:\texttt{args}(TLV bytes)
\end{itemize}
InitParams の最小例では、\texttt{0x0101}=m(u16), \texttt{0x0102}=d(u64),
\texttt{0x0103}=t(u64) を用いる。

\begin{verbatim}
Canonical example
init_params_bytes (hex):
0101020001 010208000000000000000a 0103080000000000000014

aux (hex):
0001020001 00020101 000320aaaaaaaaaaaaaaaaaaaaaaaaaaaaaaaaaaaaaaaaaaaaaaaaaaaaaaaaaaaaaaaa
00040400000001 0006200101020001010208000000000000000a0103080000000000000014
valid: types are sorted, no duplicates, length is canonical.

Non-canonical examples (invalid)
1) Length not canonical (proto_version length encoded with extra bytes):
0001 820002 0001
invalid: length is not minimal.

2) Duplicate type (tx_type appears twice):
00020101 00020102
invalid: duplicate type.
\end{verbatim}
