\section{Related Work}

\subsection{Contract Paradigms}
Bitcoin Script/UTXOは、意図的に非チューリング完全なスタック言語として設計され、予測可能性と攻撃面の抑制を優先している。
検証は局所的であり、監査対象(audit surface)が事前に見積もりやすい点が特長である。
\cite{Nakamoto2008,Okupski2016,BitcoinWiki2025}

対照的にEthereumのEVMは、アカウント型のグローバルステートを前提に高い表現力と汎用性を提供する一方、状態更新と合成可能性の拡大により意味論が複雑化し、監査負荷や脆弱性リスクが増えやすい。
\cite{Buterin2014,Wood2025}
このギャップは単なる「表現力の差」ではなく、監査対象の規模と事前見積もり可能性の差として整理できる。
\cite{Chakravarty2020}

\subsection{Limitations}
WasmやRISC-Vのような別命令セットを採用した環境は「より良いVM」というよりも、依然としてVM型の実行環境である。
Polkadot/ink!ではWasm実行環境とRISC-V/PolkaVMの併置や移行が明示されている。
\cite{Polkadot2026,Polkadot2025,Ink2025a,Ink2025b,Ink2025c}
RISC-V上でLinuxを動かすCartesi Machineも、VM型基盤として提示される。
\cite{Teixeira2018}
したがって汎用実行と状態更新、相互作用・合成可能性を中核に置く限り、監査対象の膨張、アップグレード解釈、実装差・運用差といった論点は残りうる。
\cite{Polkadot2026,Ink2025a,Ink2025b,Teixeira2018}

\paragraph{eUTXO}
中間に位置づけられるのが eUTXO であり、UTXOの意味論的な単純さを維持しつつ、validator scripts と datum/redeemer/context によって表現力を拡張する。
\cite{Chakravarty2020}
Cardano側はトランザクション検証の決定性(no surprises)を強調しており、予測可能性を保持した表現拡張として位置づけられる。
\cite{IOG2021a,IOG2021b,Cardano2025a,Cardano2025b,Chakravarty2020}

\paragraph{Szabo}
Szaboは、ネットワーク上の関係(信用・権利・支払い等)をプロトコルとして形式化し、セキュアに実装するという問題設定を提示した。
\cite{Szabo1996,Szabo1997}
この系譜は、スマートコントラクトを「汎用プログラム」ではなく「関係を固定する取引プロトコル」として捉える視点を支持する。
\cite{Szabo1996,Szabo1997}

\subsection{Gap}
既存研究は、(i) UTXO/Scriptおよびその拡張が提供する局所検証性・予測可能性と、(ii) VM型実行環境が提供する汎用性・合成可能性の間にあるトレードオフを繰り返し指摘してきた。
\cite{Nakamoto2008,Okupski2016,Chakravarty2020,Polkadot2026,Wood2025}
しかし、匿名・身元不確かな市場で頻出するエスクロー等の設計に対し、許される遷移を列挙でき、成立条件と遷移結果が検証可能に固定され、監査負荷が上限付きとなる「テンプレート型契約」として、この中間領域を体系化した枠組みは十分整理されていない。
\cite{Chakravarty2020,Szabo1996}
その結果、実務では評判や運営判断に依存する余地(カストディ集中・運用裁量)が残りうる。
本稿は、このギャップに対してテンプレート型契約の形式化と遷移仕様化を提示し、検証可能性と監査上限を両立する設計指針を明確化する。
